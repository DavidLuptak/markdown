% \iffalse
%<*driver>
\documentclass{ltxdockit}
\usepackage{btxdockit}
\usepackage[utf8]{inputenc}
\usepackage[T1]{fontenc}
\usepackage[american]{babel}
\usepackage{microtype}
\usepackage{amsmath}
\usepackage{fancyvrb}
\usepackage{hologo}
\usepackage{xcolor}
\usepackage{doc}

% Set up the style.
\emergencystretch=1em
\RecustomVerbatimEnvironment
  {Verbatim}{Verbatim}
  {gobble=2,frame=single}
\setcounter{secnumdepth}{4}
\addtokomafont{title}{\sffamily}
\addtokomafont{paragraph}{\spotcolor}
\addtokomafont{section}{\spotcolor}
\addtokomafont{subsection}{\spotcolor}
\addtokomafont{subsubsection}{\spotcolor}
\addtokomafont{descriptionlabel}{\spotcolor}
\setkomafont{caption}{\bfseries\sffamily\spotcolor}
\setkomafont{captionlabel}{\bfseries\sffamily\spotcolor}
\hypersetup{citecolor=spot}
\let\oldCodelineNo\theCodelineNo
\def\theCodelineNo{\textcolor{gray}{\oldCodelineNo}}

% Define some markup.
\let\pkg\relax % A package name
\newcommand\mdef[1]{% A TeX macro definition
  \phantomsection\label{macro:#1}\textcolor{spot}{\cs{#1}}}
\newcommand\m[1]{% A TeX macro reference
  \hyperref[macro:#1]{\textcolor{spot}{\cs{#1}}}}
\newcommand\envmdef[1]{% A TeX macro definition
  \phantomsection\label{environment:#1}\textcolor{spot}{\t`#1`}}
\newcommand\envm[1]{% A LaTeX environment reference
  \hyperref[environment:#1]{\textcolor{spot}{\t`#1`}}}
\newcommand\luamdef[1]{% A Lua object / method definition
  \phantomsection\label{lua:#1}\textcolor{spot}{\t`#1`}}
\newcommand\luam[1]{% A Lua object / method reference
  \hyperref[lua:#1]{\t`#1`}}
\def\t`#1`{% Inline code
  \textcolor{spot}{\text{\texttt{#1}}}}
\newcommand\Optitem[2][]{% An option item definition
  \phantomsection\label{opt:#2}\optitem[#1]{#2}}
\newcommand\Valitem[2][]{% A value item definition
  \phantomsection\label{opt:#2}\valitem[#1]{#2}}
\newcommand\Opt[1]{% An option / value item reference
  \hyperref[opt:#1]{\t`#1`}}
\newcommand\acro[1]{% An acronym
  \textsc{#1}}

% Set up the catcodes.
\catcode`\_=12 % We won't be typesetting math and Lua contains lots of `_`.

% Set up the title page.
\titlepage{%
  title={A Markdown Interpreter for \TeX{}},
  subtitle={},
  url={https://github.com/witiko/markdown},
  author={Vít Novotný (based on the work of John MacFarlane and Hans Hagen)},
  email={witiko@mail.muni.cz},
  revision={\input VERSION},
  date={\today}}
\CodelineIndex

% Set up the bibliography.
\usepackage{filecontents}
\begin{filecontents}{markdown.bib}
  @book{luatex16,
    author    = {{Lua\TeX{} development team}},
    title     = {Lua\TeX{} reference manual (0.95.0)},
    url       = {http://www.luatex.org/svn/trunk/manual/luatex.pdf},
    urldate   = {2016-05-12}}
  @book{latex16,
    author    = {Braams, Johannes and Carlisle, David and Jeffrey, Alan and
                 Lamport, Leslie and Mittelbach, Frank and Rowley, Chris and
                 Schöpf, Rainer},
    title     = {The \Hologo{LaTeX2e} Sources},
    date      = {2016-03-31},
    url       = {http://mirrors.ctan.org/macros/latex/base/source2e.pdf},
    urldate   = {2016-06-02}}
  @book{ierusalimschy13,
    author    = {Ierusalimschy, Roberto},
    year      = {2013},
    title     = {Programming in Lua},
    edition   = {3},
    isbn      = {978-85-903798-5-0},
    pagetotal = {xviii, 347},
    location  = {Rio de Janeiro},
    publisher = {PUC-Rio}}
  @book{knuth86,
    author    = {Knuth, Donald Ervin},
    year      = {1986},
    title     = {The \TeX{}book},
    edition   = {3},
    isbn      = {0-201-13447-0},
    pagetotal = {ix, 479},
    publisher = {Addison-Westley}}
\end{filecontents}
\usepackage[
  backend=biber,
  style=iso-numeric,
  sorting=none,
  autolang=other,
  sortlocale=auto]{biblatex}
\addbibresource{markdown.bib}

\begin{document}
  \printtitlepage
  \tableofcontents
  \DocInput{markdown.dtx}
  \printbibliography
\end{document}
%</driver>
% \fi
%
% \section{Introduction}
% This document is a reference manual for the \pkg{Markdown} package. It is
% split into three sections. This section explains the purpose and the
% background of the package and outlines its prerequisites. Section
% \ref{sec:interfaces} describes the interfaces exposed by the package along
% with usage notes and examples. It is aimed at the user of the package.
% Section \ref{sec:implementation} describes the implementation of the package.
% It is aimed at the developer of the package and the curious user.
%
% \subsection{About \pkg{Markdown}}
% The \pkg{Markdown} package provides facilities for the conversion of markdown
% markup to plain \TeX{}. These are provided both in the form of a Lua module
% and in the form of plain \TeX{}, \LaTeX{}, and \Hologo{ConTeXt} macro
% packages that enable the direct inclusion of markdown documents inside \TeX{}
% documents.
%
% Architecturally, the package consists of the \pkg{Lunamark} Lua module by
% John MacFarlane, which was slimmed down and rewritten for the needs of the
% package. On top of \pkg{Lunamark} sits code for the plain \TeX{}, \LaTeX{},
% and \Hologo{ConTeXt} formats by Vít Novotný.
%
% \iffalse
%<*lua>
% \fi
%  \begin{macrocode}
if not modules then modules = { } end modules ['markdown'] = {
    version   = "0.4.0",
    comment   = "A module for the conversion from markdown to plain TeX",
    author    = "John MacFarlane, Hans Hagen, Vít Novotný",
    copyright = "2009-2016 John MacFarlane, Hans Hagen; 2016 Vít Novotný",
    license   = "LPPL 1.3"
}
%    \end{macrocode}
%
% \subsection{Feedback}
% Please use the \pkg{markdown} project page on
% GitHub\footnote{\url{https://github.com/witiko/markdown/issues}} to report
% bugs and submit feature requests. Before making a feature request, please
% ensure that you have thoroughly studied this manual. If you do not want to
% report a bug or request a feature but are simply in need of assistance, you
% might want to consider posting your question on the \TeX-\LaTeX{} Stack
% Exchange\footnote{\url{https://tex.stackexchange.com}}.
%
% \subsection{Acknowledgements}
% I would like to thank the Faculty of Informatics at the Masaryk University in
% Brno for providing me with the opportunity to work on this package alongside
% my studies. I would also like to thank the creator of the Lunamark Lua
% module, John Macfarlane, for releasing Lunamark under a permissive license
% that enabled its inclusion into the package.
%
% The \TeX{} part of the package draws inspiration from several sources
% including the source code of \Hologo{LaTeX2e}, the \pkg{minted} package by
% Geoffrey M. Poore -- which likewise tackles the issue of interfacing with an
% external interpreter from \TeX, the \pkg{filecontents} package by Scott
% Pakin, and others.
%
% \subsection{Prerequisites}
% This section gives an overview of all resources required by the package.
%
% \subsubsection{Lua Prerequisites}\label{sec:luaprerequisites}
% The Lua part of the package requires the following Lua modules:
% \begin{description}
%   \item[\pkg{LPeg}] A pattern-matching library for the writing of recursive
%     descent parsers via the Parsing Expression Grammars (\acro{peg}s). It is
%     used by the \pkg{Lunamark} library to parse the markdown input.
%  \begin{macrocode}
local lpeg = require("lpeg")
%    \end{macrocode}
%   \item[\pkg{Selene Unicode}] A library that provides support for the
%     processing of wide strings. It is used by the \pkg{Lunamark} library to
%     cast image, link, and footnote tags to the lower case.
%  \begin{macrocode}
local unicode = require("unicode")
%    \end{macrocode}
%   \item[\pkg{MD5}] A library that provides \acro{md5} crypto functions. It is
%     used by the \pkg{Lunamark} library to compute the digest of the input for
%     caching purposes. 
%  \begin{macrocode}
local md5 = require("md5")
%    \end{macrocode}
% \end{description}
% All the abovelisted modules are statically linked into the Lua\TeX{} engine
% (see \cite[Section~3.3]{luatex16}).
%
% \iffalse
%</lua>
%<*tex>
% \fi
% \subsubsection{Plain \TeX{} Prerequisites}\label{sec:texprerequisites}
% The plain \TeX{} part of the package requires the following Lua module:
% \begin{description}
%   \item[\pkg{Lua File System}] A library that provides access to the
%     filesystem via \acro{os}-specific syscalls. It is used by the plain
%     \TeX{} code to create the cache directory specified by the
%     \m{markdownOptionCacheDir} macro before interfacing with the
%     \pkg{Lunamark} library.
%
%     The plain \TeX{} code makes use of the \luam{isDir} method that was added
%     to the module by the Lua\TeX{} engine developers (see
%     \cite[Section~3.2]{luatex16}). This method is not present in the base
%     library.
% \end{description}
% The \pkg{Lua File System} module is statically linked into the Lua\TeX{}
% engine (see \cite[Section~3.3]{luatex16}).
%
% The plain \TeX{} part of the package also requires that the plain \TeX{}
% format (or its superset) is loaded and that either the Lua\TeX{}
% \m{directlua} primitive or the shell access file stream 18 is available.
%
% \iffalse
%</tex>
%<*latex>
% \fi
% \subsubsection{\LaTeX{} Prerequisites}\label{sec:latexprerequisites}
% The \LaTeX{} part of the package requires that the \Hologo{LaTeX2e} format is
% loaded and also, since it uses the plain \TeX{} implementation, all the plain
% \TeX{} prerequisites (see Section \ref{sec:texprerequisites}).
%  \begin{macrocode}
\NeedsTeXFormat{LaTeX2e}%
%    \end{macrocode}
%
% The following \Hologo{LaTeX2e} packages are also required:
% \begin{description}
%   \item[\pkg{keyval}] A package that enables the creation of parameter sets.
%     This package is used to provide the \m{markdownSetup} macro, the package
%     options processing, as well as the parameters of the \envm{markdown*}
%     \LaTeX{} environment.
%   \item[\pkg{url}] A package that provides the \m{url} macro for the
%     typesetting of \acro{url}s. It is used to provide the default token renderer
%     prototype (see Section \ref{sec:texrendererprototypes}) for links.
%   \item[\pkg{graphicx}] A package that provides the \m{includegraphics} macro for
%     the typesetting of images. It is used to provide the corresponding
%     default token renderer prototype (see Section
%     \ref{sec:texrendererprototypes}).
%   \item[\pkg{paralist}] A package that provides the \envm{compactitem},
%     \envm{compactenum}, and \envm{compactdesc} macros for the
%     typesetting of tight bulleted lists, ordered lists, and definition lists.
%     It is used to provide the corresponding default token renderer prototypes
%     (see Section \ref{sec:texrendererprototypes}).
%   \item[\pkg{fancyvrb}] A package that provides the \m{VerbatimInput} macros
%     for the verbatim inclusion of files containing code.  It is used to
%     provide the corresponding default token renderer prototype (see Section
%     \ref{sec:texrendererprototypes}).
% \end{description}
%
% \iffalse
%</latex>
%<*context>
% \fi
% \subsubsection{\Hologo{ConTeXt} prerequisites}
% The \Hologo{ConTeXt} part of the package requires that either the Mark II or
% the Mark IV format is loaded and also, since it uses the plain \TeX{}
% implementation, all the plain \TeX{} prerequisites (see Section
% \ref{sec:texprerequisites}).
%
% \section{User Guide}\label{sec:interfaces}
% This part of the manual describes the interfaces exposed by the package
% along with usage notes and examples. It is aimed at the user of the package.
%
% Since neither \TeX{} nor Lua provide interfaces as a language construct, the
% separation to interfaces and implementations is purely abstract. It serves as
% a means of structuring this manual and as a promise to the user that if they
% only access the package through the interfaces, the future versions of the
% package should remain backwards compatible.
%
% \iffalse
%</context>
%<*lua>
% \fi
% \subsection{Lua Interface}\label{sec:luainterface}
% The Lua interface provides the conversion from markdown to plain \TeX{}.
% This interface is used by the plain \TeX{} implementation (see Section
% \ref{sec:teximplementation}) and will be of interest to the developers of
% other packages and Lua modules.
%
% The Lua interface is implemented by the \t`markdown` Lua module.
%
%  \begin{macrocode}
local M = {}
%    \end{macrocode}
%
% \subsubsection{Conversion from Markdown to Plain \TeX{}}
% \label{sec:luaconversion}
% The Lua interface exposes the \luamdef{new}\t`(options)` method.  This
% method creates converter functions that perform the conversion from markdown
% to plain \TeX{} according to the table \t`options` that contains options
% recognized by the Lua interface.  (see Section \ref{sec:luaoptions}). The
% \t`options` parameter is optional; when unspecified, the behaviour will be
% the same as if \t`options` were an empty table.
%  \begin{macrocode}
M.new = {}
%    \end{macrocode}
%
% The following example Lua code converts the markdown string \t`_Hello
% world!_` to a \TeX{} output using the default options and prints the \TeX{}
% output:
% \begin{Verbatim}
% local md = require("markdown")
% local convert = md.new()
% print(convert("_Hello world!_"))
% \end{Verbatim}
%
% \subsubsection{Options}\label{sec:luaoptions}
% The Lua interface recognizes the following options. When unspecified, the
% value of a key is taken from the \luamdef{defaultOptions} table.
%  \begin{macrocode}
local defaultOptions = {}
%    \end{macrocode}
% \begin{optionlist}
%   \Optitem[false]{blankBeforeBlockquote}{\opt{true}, \opt{false}}
%     \begin{valuelist}
%       \item[true] Require a blank line between a paragraph and the following
%         blockquote.
%       \item[false] Do not require a blank line between a paragraph and the
%         following blockquote.
%     \end{valuelist}
%  \begin{macrocode}
defaultOptions.blankBeforeBlockquote = false
%    \end{macrocode}
%
%   \Optitem[false]{blankBeforeHeading}{\opt{true}, \opt{false}}
%     \begin{valuelist}
%       \item[true] Require a blank line between a paragraph and the following
%         header.
%       \item[false] Do not require a blank line between a paragraph and the
%         following header.
%     \end{valuelist}
%  \begin{macrocode}
defaultOptions.blankBeforeHeading = false
%    \end{macrocode}
%
%   \Valitem[.]{cacheDir}{directory}
%     The path to the directory containing auxiliary cache files.
% 
%     When iteratively writing and typesetting a markdown document, the cache
%     files are going to accumulate over time. You are advised to clean the
%     cache directory every now and then, or to set it to a temporary filesystem
%     (such as \t`/tmp` on \acro{un*x} systems), which gets periodically
%     emptied.
%  \begin{macrocode}
defaultOptions.cacheDir = "."
%    \end{macrocode}
%
%   \Optitem[false]{definitionLists}{\opt{true}, \opt{false}}
%     \begin{valuelist}
%       \item[true] Enable the pandoc definition list syntax extension:
%         \begin{Verbatim}
% Term 1
% 
% :   Definition 1
% 
% Term 2 with *inline markup*
% 
% :   Definition 2
% 
%         { some code, part of Definition 2 }
% 
%     Third paragraph of definition 2.
%         \end{Verbatim}
%       \item[false] Disable the pandoc definition list syntax extension.
%     \end{valuelist}
%  \begin{macrocode}
defaultOptions.definitionLists = false
%    \end{macrocode}
% 
%   \Optitem[false]{hashEnumerators}{\opt{true}, \opt{false}}
%     \begin{valuelist}
%       \item[true] Enable the use of hash symbols (\t`\#`) as ordered item
%         list markers.
%       \item[false] Disable the use of hash symbols (\t`\#`) as ordered item
%         list markers.
%     \end{valuelist}
%  \begin{macrocode}
defaultOptions.hashEnumerators = false
%    \end{macrocode}
%
%   \Optitem[false]{hybrid}{\opt{true}, \opt{false}}
%     \begin{valuelist}
%       \item[true] Disable the escaping of special plain \TeX{} characters,
%         which makes it possible to intersperse your markdown markup with
%         \TeX{} code. The intended usage is in documents prepared manually by
%         a human author.  In such documents, it can often be desirable to mix
%         \TeX{} and markdown markup freely.
%
%       \item[false] Enable the escaping of special plain \TeX{} characters
%         outside verbatim environments, so that they are not interpretted by
%         \TeX{}. This is encouraged when typesetting automatically generated
%         content or markdown documents that were not prepared with this
%         package in mind.
%     \end{valuelist}
%  \begin{macrocode}
defaultOptions.hybrid = false
%    \end{macrocode}
%
%   \Optitem[false]{footnotes}{\opt{true}, \opt{false}}
%     \begin{valuelist}
%       \item[true] Enable the pandoc footnote syntax extension:
%         \begin{Verbatim}
% Here is a footnote reference,[^1] and another.[^longnote]
% 
% [^1]: Here is the footnote.
% 
% [^longnote]: Here's one with multiple blocks.
% 
%     Subsequent paragraphs are indented to show that they
% belong to the previous footnote.
% 
%         { some.code }
% 
%     The whole paragraph can be indented, or just the
%     first line.  In this way, multi-paragraph footnotes
%     work like multi-paragraph list items.
% 
% This paragraph won't be part of the note, because it
% isn't indented.
%         \end{Verbatim}
%       \item[false] Disable the pandoc footnote syntax extension.
%     \end{valuelist}
%  \begin{macrocode}
defaultOptions.footnotes = false
%    \end{macrocode}
%
%   \Optitem[false]{preserveTabs}{\opt{true}, \opt{false}}
%     \begin{valuelist}
%       \item[true] Preserve all tabs in the input.
%       \item[false] Convert any tabs in the input to spaces.
%     \end{valuelist}
%  \begin{macrocode}
defaultOptions.preserveTabs = false
%    \end{macrocode}
%
%   \Optitem[false]{smartEllipses}{\opt{true}, \opt{false}}
%     \begin{valuelist}
%       \item[true] Convert any ellipses in the input to the
%         \m{markdownRendererEllipsis} \TeX{} macro.
%       \item[false] Preserve all ellipses in the input.
%     \end{valuelist}
%  \begin{macrocode}
defaultOptions.smartEllipses = false
%    \end{macrocode}
%
%   \Optitem[true]{startNumber}{\opt{true}, \opt{false}}
%     \begin{valuelist}
%       \item[true] Make the number in the first item in ordered lists
%         significant. The item numbers will be passed to the
%         \m{markdownRendererOlItemWithNumber} \TeX{} macro.
%       \item[false] Ignore the number in the items of ordered lists. Each
%         item will only produce a \m{markdownRendererOlItem} \TeX{} macro.
%     \end{valuelist}
%  \begin{macrocode}
defaultOptions.startNumber = true
%    \end{macrocode}
% \end{optionlist}
% 
% \iffalse
%</lua>
%<*tex>
% \fi\subsection{Plain \TeX{} Interface}\label{sec:texinterface}
% The plain \TeX{} interface provides macros for the typesetting of markdown input
% from within plain \TeX{}, for setting the Lua interface options (see Section
% \ref{sec:luaoptions}) used during the conversion from markdown to plain
% \TeX{}, and for changing the way markdown the tokens are rendered.
%  \begin{macrocode}
\def\markdownVersion{2016/06/03}%
%    \end{macrocode}
%
% The plain \TeX{} interface is implemented by the \t`markdown.tex` file that
% can be loaded as follows:
% \begin{Verbatim}
% \input markdown
% \end{Verbatim}
% It is expected that the special plain \TeX{} characters have the expected
% category codes, when \m{input}ting the file.
%
% \subsubsection{Typesetting Markdown}\label{sec:textypesetting}
% The interface exposes the \mdef{markdownBegin}, \mdef{markdownEnd}, and
% \mdef{markdownInput} macros.
%
% The \m{markdownBegin} macro marks the beginning of a markdown document
% fragment and the \m{markdownEnd} macro marks its end.
%  \begin{macrocode}
\let\markdownBegin\relax
\let\markdownEnd\relax
%    \end{macrocode}
% You may prepend your own code to the \m{markdownBegin} macro and redefine the
% \m{markdownEnd} macro to produce special effects before and after the
% markdown block.
%
% There are several limitations to the macros you need to be aware of.
% The first limitation concerns the \m{markdownEnd} macro, which must be
% visible directly from the input line buffer (it may not be produced as a
% result of input expansion). Otherwise, it will not be recognized as the end
% of the markdown string otherwise. As a corrolary, the \m{markdownEnd} string
% may not appear anywhere inside the markdown input.
%
% Another limitation concerns spaces at the right end of an input line. In
% markdown, these are used to produce a forced line break. However, any such
% spaces are removed before the lines enter the input buffer of \TeX{} (see
% \cite[p.~46]{knuth86}). As a corrolary, the \m{markdownBegin} macro also
% ignores them.
%
% The \m{markdownBegin} and \m{markdownEnd} macros will also consume the rest
% of the lines at which they appear.  In the following example plain \TeX{}
% code, the characters \t`c`, \t`e`, and \t`f` will not appear in the output.
% \begin{Verbatim}
% \input markdown
% a
% b \markdownBegin c 
% d 
% e \markdownEnd   f
% g 
% \bye
% \end{Verbatim}
%
% Note that you may also not nest the \m{markdownBegin} and \m{markdownEnd}
% macros.
%
% The following example plain \TeX{} code showcases the usage of the
% \m{markdownBegin} and \m{markdownEnd} macros:
% \begin{Verbatim}
% \input markdown
% \markdownBegin
% _Hello_ **world** ...
% \markdownEnd
% \bye
% \end{Verbatim}
%
% The \m{markdownInput} macro accepts a single parameter containing the
% filename of a markdown document and expands to the result of the conversion
% of the input markdown document to plain \TeX{}.
%  \begin{macrocode}
\let\markdownInput\relax
%    \end{macrocode}
% This macro is not subject to the abovelisted limitations of the
% \m{markdownBegin} and \m{markdownEnd} macros.
%
% The following example plain \TeX{} code showcases the usage of the
% \m{markdownInput} macro:
% \begin{Verbatim}
% \input markdown
% \markdownInput{hello.md}
% \bye
% \end{Verbatim}
%
% \subsubsection{Options}\label{sec:texoptions}
% The plain \TeX{} options are represented by \TeX{} macros. Some of them map
% directly to the options recognized by the Lua interface (see Section
% \ref{sec:luaoptions}), while some of them are specific to the plain \TeX{}
% interface.
%
% \paragraph{File and directory names}
% The \mdef{markdownOptionHelperScriptFileName} macro sets the filename of the
% helper Lua script file that is created during the conversion from markdown to
% plain \TeX{} in \TeX{} engines without the \m{directlua} primitive. It
% defaults to \m{jobname}\t`.markdown.lua`, where \m{jobname} is the base name
% of the document being typeset.
%
% The expansion of this macro must not contain quotation marks (\t`"`) or
% backslash symbols (\t`\textbackslash`). Mind that \TeX{} engines tend to
% put quotation marks around \m{jobname}, when it contains spaces.
%  \begin{macrocode}
\def\markdownOptionHelperScriptFileName{\jobname.markdown.lua}%
%    \end{macrocode}
%
% The \mdef{markdownOptionInputTempFileName} macro sets the filename of the
% temporary input file that is created during the conversion from markdown to
% plain \TeX{} in \TeX{} engines without the \m{directlua} primitive. It
% defaults to \m{jobname}\t`.markdown.out`. The same limitations as in
% the case of the \m{markdownOptionHelperScriptFileName} macro apply here.
%  \begin{macrocode}
\def\markdownOptionInputTempFileName{\jobname.markdown.in}%
%    \end{macrocode}
%
% The \mdef{markdownOptionOutputTempFileName} macro sets the filename of the
% temporary output file that is created during the conversion from markdown to
% plain \TeX{} in \TeX{} engines without the \m{directlua} primitive. It
% defaults to \m{jobname}\t`.markdown.out`. The same limitations apply here as
% in the case of the \m{markdownOptionHelperScriptFileName} macro.
%  \begin{macrocode}
\def\markdownOptionOutputTempFileName{\jobname.markdown.out}%
%    \end{macrocode}
%
% The \mdef{markdownOptionCacheDir} macro corresponds to the Lua interface
% \Opt{cacheDir} option that sets the name of the directory that will contain
% the produced cache files. The option defaults to \t`_markdown-`\m{jobname},
% which is a similar naming scheme to the one used by the \pkg{minted} \LaTeX{}
% package. The same limitations apply here as in the case of the
% \m{markdownOptionHelperScriptFileName} macro.
%  \begin{macrocode}
\def\markdownOptionCacheDir{_markdown-\jobname}%
%    \end{macrocode}
%
% \paragraph{Lua Interface Options}
% The following macros map directly to the options recognized by the Lua
% interface (see Section \ref{sec:luaoptions}) and are not processed by the
% plain \TeX{} implementation, only passed along to Lua. They are undefined, which
% makes them fall back to the default values provided by the Lua interface.
%  \begin{macrocode}
\let\markdownOptionBlankBeforeBlockquote\undefined
\let\markdownOptionBlankBeforeHeading\undefined
\let\markdownOptionDefinitionLists\undefined
\let\markdownOptionHashEnumerator\undefined
\let\markdownOptionHybrid\undefined
\let\markdownOptionFootnotes\undefined
\let\markdownOptionPreserveTabs\undefined
\let\markdownOptionSmartEllipses\undefined
\let\markdownOptionStartNumber\undefined
\let\markdownOptionVerbatim\undefined
%    \end{macrocode}
%
% \subsubsection{Token Renderers}\label{sec:texrenderersuser}
% The following \TeX{} macros may occur inside the output of the
% converter functions exposed by the Lua interface (see Section
% \ref{sec:luaconversion}) and represent the parsed markdown tokens. These
% macros are intended to be redefined by the user who is typesetting a
% document. By default, they point to the corresponding prototypes (see Section
% \ref{sec:texrendererprototypes}).
%
% \paragraph{Line Break Renderer}
% The \mdef{markdownRendererLineBreak} macro represents a forced line break.
% The macro receives no arguments.
%  \begin{macrocode}
\def\markdownRendererLineBreak{%
  \markdownRendererLineBreakPrototype}%
%    \end{macrocode}
%
% \paragraph{Ellipsis Renderer}
% The \mdef{markdownRendererEllipsis} macro replaces any occurance of ASCII
% ellipses in the input text. This macro will only be produced, when the
% \Opt{smartEllipses} option is \t`true`.  The macro receives no arguments.
%  \begin{macrocode}
\def\markdownRendererEllipsis{%
  \markdownRendererEllipsisPrototype}%
%    \end{macrocode}
%
% \paragraph{Code Span Renderer}
% The \mdef{markdownRendererCodeSpan} macro represents inlined code span in the
% input text. It receives a single argument that corresponds to the inlined
% code span.
%  \begin{macrocode}
\def\markdownRendererCodeSpan{%
  \markdownRendererCodeSpanPrototype}%
%    \end{macrocode}
%
% \paragraph{Link Renderer}
% The \mdef{markdownRendererLink} macro represents a hyperlink. It receives
% three arguments: the label, the \acro{uri}, and the title of the link.
%  \begin{macrocode}
\def\markdownRendererLink{%
  \markdownRendererLinkPrototype}%
%    \end{macrocode}
%
% \paragraph{Image Renderer}
% The \mdef{markdownRendererImage} macro represents an image. It receives three
% arguments: the label, the \acro{url}, and the title of the image.
%  \begin{macrocode}
\def\markdownRendererImage{%
  \markdownRendererImagePrototype}%
%    \end{macrocode}
%
% \paragraph{Bullet List Renderers}
% The \mdef{markdownRendererUlBegin} macro represents the beginning of a
% bulleted list that contains an item with several paragraphs of text (the
% list is not tight). The macro receives no arguments.
%  \begin{macrocode}
\def\markdownRendererUlBegin{%
  \markdownRendererUlBeginPrototype}%
%    \end{macrocode}
%
% The \mdef{markdownRendererUlBeginTight} macro represents the beginning of a
% bulleted list that contains no item with several paragraphs of text (the list
% is tight). The macro receives no arguments.
%  \begin{macrocode}
\def\markdownRendererUlBeginTight{%
  \markdownRendererUlBeginTightPrototype}%
%    \end{macrocode}
%
% The \mdef{markdownRendererUlItem} macro represents an item in a bulleted
% list.
% The macro receives no arguments.
%  \begin{macrocode}
\def\markdownRendererUlItem{%
  \markdownRendererUlItemPrototype}%
%    \end{macrocode}
%
% The \mdef{markdownRendererUlEnd} macro represents the end of a bulleted list
% that contains an item with several paragraphs of text (the list is not
% tight). The macro receives no arguments.
%  \begin{macrocode}
\def\markdownRendererUlEnd{%
  \markdownRendererUlEndPrototype}%
%    \end{macrocode}
%
% The \mdef{markdownRendererUlEndTight} macro represents the end of a bulleted
% list that contains no item with several paragraphs of text (the list is
% tight). The macro receives no arguments.
%  \begin{macrocode}
\def\markdownRendererUlEndTight{%
  \markdownRendererUlEndTightPrototype}%
%    \end{macrocode}
%
% \paragraph{Ordered List Renderers}
% The \mdef{markdownRendererOlBegin} macro represents the beginning of an
% ordered list that contains an item with several paragraphs of text (the
% list is not tight). The macro receives no arguments.
%  \begin{macrocode}
\def\markdownRendererOlBegin{%
  \markdownRendererOlBeginPrototype}%
%    \end{macrocode}
%
% The \mdef{markdownRendererOlBeginTight} macro represents the beginning of an
% ordered list that contains no item with several paragraphs of text (the
% list is tight). The macro receives no arguments.
%  \begin{macrocode}
\def\markdownRendererOlBeginTight{%
  \markdownRendererOlBeginTightPrototype}%
%    \end{macrocode}
%
% The \mdef{markdownRendererOlItem} macro represents an item in an ordered list.
% This macro will only be produced, when the \Opt{startNumber} option is
% \t`false`.  The macro receives no arguments.
%  \begin{macrocode}
\def\markdownRendererOlItem{%
  \markdownRendererOlItemPrototype}%
%    \end{macrocode}
%
% The \mdef{markdownRendererOlItemWithNumber} macro represents an item in an
% ordered list.  This macro will only be produced, when the \Opt{startNumber}
% option is \t`true`.  The macro receives no arguments.
%  \begin{macrocode}
\def\markdownRendererOlItemWithNumber{%
  \markdownRendererOlItemWithNumberPrototype}%
%    \end{macrocode}
%
% The \mdef{markdownRendererOlEnd} macro represents the end of an ordered list
% that contains an item with several paragraphs of text (the list is not
% tight). The macro receives no arguments.
%  \begin{macrocode}
\def\markdownRendererOlEnd{%
  \markdownRendererOlEndPrototype}%
%    \end{macrocode}
%
% The \mdef{markdownRendererOlEndTight} macro represents the end of an ordered
% list that contains no item with several paragraphs of text (the list is
% tight). The macro receives no arguments.
%  \begin{macrocode}
\def\markdownRendererOlEndTight{%
  \markdownRendererOlEndTightPrototype}%
%    \end{macrocode}
%
% \paragraph{Definition List Renderers}
% The following macros are only produces, when the \Opt{definitionLists} option
% is \t`true`.
%
% The \mdef{markdownRendererDlBegin} macro represents the beginning of a
% definition list that contains an item with several paragraphs of text (the
% list is not tight). The macro receives no arguments.
%  \begin{macrocode}
\def\markdownRendererDlBegin{%
  \markdownRendererDlBeginPrototype}%
%    \end{macrocode}
%
% The \mdef{markdownRendererDlBeginTight} macro represents the beginning of a
% definition list that contains an item with several paragraphs of text (the
% list is not tight). The macro receives no arguments.
%  \begin{macrocode}
\def\markdownRendererDlBeginTight{%
  \markdownRendererDlBeginTightPrototype}%
%    \end{macrocode}
%
% The \mdef{markdownRendererDlItem} macro represents an item in a definition
% list. The macro receives a single argument that corresponds to the term being
% defined.
%  \begin{macrocode}
\def\markdownRendererDlItem{%
  \markdownRendererDlItemPrototype}%
%    \end{macrocode}
%
% The \mdef{markdownRendererDlEnd} macro represents the end of a definition
% list that contains an item with several paragraphs of text (the list is not
% tight). The macro receives no arguments.
%  \begin{macrocode}
\def\markdownRendererDlEnd{%
  \markdownRendererDlEndPrototype}%
%    \end{macrocode}
%
% The \mdef{markdownRendererDlEndTight} macro represents the end of a
% definition list that contains no item with several paragraphs of text (the
% list is tight). The macro receives no arguments.
%  \begin{macrocode}
\def\markdownRendererDlEndTight{%
  \markdownRendererDlEndTightPrototype}%
%    \end{macrocode}
%
% \paragraph{Emphasis Renderers}
% The \mdef{markdownRendererEmphasis} macro represents an emphasized span of
% text. The macro receives a single argument that corresponds to the emphasized
% span of text.
%  \begin{macrocode}
\def\markdownRendererEmphasis{%
  \markdownRendererEmphasisPrototype}%
%    \end{macrocode}
%
% The \mdef{markdownRendererStrongEmphasis} macro represents a strongly
% emphasized span of text. The macro receives a single argument that
% corresponds to the emphasized span of text.
%  \begin{macrocode}
\def\markdownRendererStrongEmphasis{%
  \markdownRendererStrongEmphasisPrototype}%
%    \end{macrocode}
%
% \paragraph{Block Quote Renderers}
% The \mdef{markdownRendererBlockQuoteBegin} macro represents the beginning of
% a block quote. The macro receives no arguments.
%  \begin{macrocode}
\def\markdownRendererBlockQuoteBegin{%
  \markdownRendererBlockQuoteBeginPrototype}%
%    \end{macrocode}
%
% The \mdef{markdownRendererBlockQuoteEnd} macro represents the end of a block
% quote. The macro receives no arguments.
%  \begin{macrocode}
\def\markdownRendererBlockQuoteEnd{%
  \markdownRendererBlockQuoteEndPrototype}%
%    \end{macrocode}
%
% \paragraph{Code Block Renderer}
% The \mdef{markdownRendererInputVerbatim} macro represents a code
% block. The macro receives a single argument that corresponds to the
% filename of a file contaning the code block to input.
%  \begin{macrocode}
\def\markdownRendererInputVerbatim{%
  \markdownRendererInputVerbatimPrototype}%
%    \end{macrocode}
%
% \paragraph{Heading Renderers}
% The \mdef{markdownRendererHeadingOne} macro represents a first level heading.
% The macro receives a single argument that corresponds to the heading text.
%  \begin{macrocode}
\def\markdownRendererHeadingOne{%
  \markdownRendererHeadingOnePrototype}%
%    \end{macrocode}
%
% The \mdef{markdownRendererHeadingTwo} macro represents a second level
% heading. The macro receives a single argument that corresponds to the heading
% text.
%  \begin{macrocode}
\def\markdownRendererHeadingTwo{%
  \markdownRendererHeadingTwoPrototype}%
%    \end{macrocode}
%
% The \mdef{markdownRendererHeadingThree} macro represents a third level
% heading. The macro receives a single argument that corresponds to the heading
% text.
%  \begin{macrocode}
\def\markdownRendererHeadingThree{%
  \markdownRendererHeadingThreePrototype}%
%    \end{macrocode}
%
% The \mdef{markdownRendererHeadingFour} macro represents a fourth level
% heading. The macro receives a single argument that corresponds to the heading
% text.
%  \begin{macrocode}
\def\markdownRendererHeadingFour{%
  \markdownRendererHeadingFourPrototype}%
%    \end{macrocode}
%
% The \mdef{markdownRendererHeadingFive} macro represents a fifth level
% heading. The macro receives a single argument that corresponds to the heading
% text.
%  \begin{macrocode}
\def\markdownRendererHeadingFive{%
  \markdownRendererHeadingFivePrototype}%
%    \end{macrocode}
%
% The \mdef{markdownRendererHeadingSix} macro represents a sixth level
% heading. The macro receives a single argument that corresponds to the heading
% text.
%  \begin{macrocode}
\def\markdownRendererHeadingSix{%
  \markdownRendererHeadingSixPrototype}%
%    \end{macrocode}
%
% \paragraph{Horizontal Rule Renderer}
% The \mdef{markdownRendererHorizontalRule} macro represents a horizontal rule.
% The macro receives no arguments.
%  \begin{macrocode}
\def\markdownRendererHorizontalRule{%
  \markdownRendererHorizontalRulePrototype}%
%    \end{macrocode}
%
% \paragraph{Footnote Renderer}
% The \mdef{markdownRendererFootnote} macro represents a footnote. This macro
% will only be produced, when the \Opt{footnotes} option is \t`true`.  The
% macro receives a single argument that corresponds to the footnote text.
%  \begin{macrocode}
\def\markdownRendererFootnote{%
  \markdownRendererFootnotePrototype}%
%    \end{macrocode}
%
% \subsubsection{Token Renderer Prototypes}\label{sec:texrendererprototypes}
% The following \TeX{} macros provide definitions for the token renderers (see
% Section \ref{sec:texrenderersuser}) that have not been redefined by the
% user. These macros are intended to be redefined by macro package authors
% who wish to provide sensible default token renderers. They are also redefined
% by the \LaTeX{} and \Hologo{ConTeXt} implementations (see sections
% \ref{sec:lateximplementation} and \ref{sec:contextimplementation}).
%  \begin{macrocode}
\def\markdownRendererLineBreakPrototype{}%
\def\markdownRendererEllipsisPrototype{}%
\long\def\markdownRendererCodeSpanPrototype#1{}%
\long\def\markdownRendererLinkPrototype#1#2#3{}%
\long\def\markdownRendererImagePrototype#1#2#3{}%
\def\markdownRendererUlBeginPrototype{}%
\def\markdownRendererUlBeginTightPrototype{}%
\def\markdownRendererUlItemPrototype{}%
\def\markdownRendererUlEndPrototype{}%
\def\markdownRendererUlEndTightPrototype{}%
\def\markdownRendererOlBeginPrototype{}%
\def\markdownRendererOlBeginTightPrototype{}%
\def\markdownRendererOlItemPrototype{}%
\long\def\markdownRendererOlItemWithNumberPrototype#1{}%
\def\markdownRendererOlEndPrototype{}%
\def\markdownRendererOlEndTightPrototype{}%
\def\markdownRendererDlBeginPrototype{}%
\def\markdownRendererDlBeginTightPrototype{}%
\long\def\markdownRendererDlItemPrototype#1{}%
\def\markdownRendererDlEndPrototype{}%
\def\markdownRendererDlEndTightPrototype{}%
\long\def\markdownRendererEmphasisPrototype#1{}%
\long\def\markdownRendererStrongEmphasisPrototype#1{}%
\def\markdownRendererBlockQuoteBeginPrototype{}%
\def\markdownRendererBlockQuoteEndPrototype{}%
\long\def\markdownRendererInputVerbatimPrototype#1{}%
\long\def\markdownRendererHeadingOnePrototype#1{}%
\long\def\markdownRendererHeadingTwoPrototype#1{}%
\long\def\markdownRendererHeadingThreePrototype#1{}%
\long\def\markdownRendererHeadingFourPrototype#1{}%
\long\def\markdownRendererHeadingFivePrototype#1{}%
\long\def\markdownRendererHeadingSixPrototype#1{}%
\def\markdownRendererHorizontalRulePrototype{}%
\long\def\markdownRendererFootnotePrototype#1{}%
%    \end{macrocode}
%
% \subsubsection{Logging Facilities}
% The \mdef{markdownInfo}, \mdef{markdownWarning}, and
% \mdef{markdownError} macros provide access to logging to the rest of
% the macros. Their first argument specifies the text of the info, warning, or
% error message.
%  \begin{macrocode}
\def\markdownInfo#1{}%
\def\markdownWarning#1{}%
%    \end{macrocode}
% The \m{markdownError} macro receives a second argument that provides a help
% text suggesting a remedy to the error.
%  \begin{macrocode}
\def\markdownError#1{}%
%    \end{macrocode}
% You may redefine these macros to redirect and process the info, warning, and
% error messages.
%
% \subsubsection{Miscellanea}
% The \mdef{markdownLuaRegisterIBCallback} and
% \mdef{markdownLuaUnregisterIBCallback} macros specify the Lua code for
% registering and unregistering a callback for changing the contents of the
% line input buffer before a \TeX{} engine that supports direct Lua access via
% the \m{directlua} macro starts looking at it. The first argument of the
% \m{markdownLuaRegisterIBCallback} macro corresponds to the callback function
% being registered.
%
% Local members defined within \m{markdownLuaRegisterIBCallback} are guaranteed
% to be visible from \m{markdownLuaUnregisterIBCallback} and the execution of
% the two macros alternates, so it is not necessary to consider the case, when
% one of the macros is called twice in a row.
%  \begin{macrocode}
\def\markdownLuaRegisterIBCallback#1{%
  local old_callback = callback.find("process_input_buffer")
  callback.register("process_input_buffer", #1)}%
\def\markdownLuaUnregisterIBCallback{%
  callback.register("process_input_buffer", old_callback)}%
%    \end{macrocode}
%
% The \mdef{markdownMakeOther} macro is used by the package, when a \TeX{}
% engine that does not support direct Lua access is starting to buffer a text.
% The plain \TeX{} implementation changes the category code of plain \TeX{}
% special characters to other, but there may be other active characters that
% may break the output. This macro should temporarily change the category of
% these to \emph{other}.
%  \begin{macrocode}
\let\markdownMakeOther\relax
%    \end{macrocode}
%
% The \mdef{markdownReadAndConvert} macro implements the \m{markdownBegin}
% macro. The first argument specifies the token sequence that will terminate
% the markdown input (\m{markdownEnd} in the instance of the \m{markdownBegin}
% macro) when the plain \TeX{} special characters have had their category
% changed to \emph{other}. The second argument specifies the token sequence
% that will actually be inserted into the document, when the ending token
% sequence has been found.
%  \begin{macrocode}
\let\markdownReadAndConvert\relax
\begingroup
%    \end{macrocode}
% Locally swap the category code of the backslash symbol (\t`\textbackslash`)
% with the pipe symbol (\t`|`). This is required in order that all the special
% symbols in the first argument of the \t`markdownReadAndConvert` macro have
% the category code \emph{other}.
%  \begin{macrocode}
  \catcode`\|=0\catcode`\\=12%
  |gdef|markdownBegin{%
    |markdownReadAndConvert{\markdownEnd}%
                           {|markdownEnd}}%
|endgroup
%    \end{macrocode}
% The macro is exposed in the interface, so that the user can create their own
% markdown environments. Due to the way the arguments are passed to Lua (see
% Section \ref{sec:directlua}), the first argument may not contain the
% string \t`]]` (regardless of the category code of the bracket symbol (\t`]`)).
%
% \iffalse
%</tex>
%<*latex>
% \fi\subsection{\LaTeX{} Interface}\label{sec:latexinterface}
% The \LaTeX{} interface provides \LaTeX{} environments for the typesetting of
% markdown input from within \LaTeX{}, facilities for setting Lua interface
% options (see Section \ref{sec:luaoptions}) used during the conversion from
% markdown to plain \TeX{}, and facilities for changing the way markdown tokens
% are rendered. The rest of the interface is inherited from the plain \TeX{}
% interface (see Section \ref{sec:texinterface}).
%
% The \LaTeX{} interface is implemented by the \t`markdown.sty` file, which
% can be loaded from the \LaTeX{} document preamble as follows:
% \begin{Verbatim}[commandchars=\\\{\}]
% \textbackslash{}usepackage[\textrm{\meta{options}}]\{markdown\}
% \end{Verbatim}
% where \meta{options} are the \LaTeX{} interface options (see Section
% \ref{sec:latexoptions}).
%
% \subsubsection{Typesetting Markdown}
% The interface exposes the \envmdef{markdown} and \envmdef{markdown*}
% \LaTeX{} environments, and redefines the \m{markdownInput} command.
%
% The \envm{markdown} and \envm{markdown*} \LaTeX{} environments are used to
% typeset markdown document fragments. The starred version of the
% \envm{markdown} environment accepts \LaTeX{} interface options (see
% Section \ref{sec:latexoptions}) as its only argument. These options will
% only influnce this markdown document fragment.
%  \begin{macrocode}
\newenvironment{markdown}\relax\relax
\newenvironment{markdown*}[1]\relax\relax
%    \end{macrocode}
% You may prepend your own code to the \m{markdown} macro and append your own
% code to the \m{endmarkdown} macro to produce special effects before and after
% the \envm{markdown} \LaTeX{} environment (and likewise for the starred
% version).
%
% Note that the \envm{markdown} and \envm{markdown*} \LaTeX{} environments are
% subject to the same limitations as the \m{markdownBegin} and \m{markdownEnd}
% macros exposed by the plain \TeX{} interface.
%
% The following example \LaTeX{} code showcases the usage of the
% \envm{markdown} and \envm{markdown*} environments:
% \begin{Verbatim}
% \documentclass{article}            \documentclass{article}
% \usepackage{markdown}              \usepackage{markdown}
% \begin{document}                   \begin{document}
% % ...                              % ...
% \begin{markdown}                   \begin{markdown*}{smartEllipses}
% _Hello_ **world** ...              _Hello_ **world** ...
% \end{markdown}                     \end{markdown*}
% % ...                              % ...
% \end{document}                     \end{document}
% \end{Verbatim}
%
% The \m{markdownInput} macro accepts a single mandatory parameter containing
% the filename of a markdown document and expands to the result of the
% conversion of the input markdown document to plain \TeX{}.  Unlike the
% \m{markdownInput} macro provided by the plain \TeX{} interface, this macro
% also accepts \LaTeX{} interface options (see Section \ref{sec:latexoptions})
% as its optional argument. These options will only influnce this markdown
% document.
%
% The following example \LaTeX{} code showcases the usage of the
% \m{markdownInput} macro:
% \begin{Verbatim}
% \documentclass{article}
% \usepackage{markdown}
% \begin{document}
% % ...
% \markdownInput[smartEllipses]{hello.md}
% % ...
% \end{document}
% \end{Verbatim}
%
% \subsubsection{Options}\label{sec:latexoptions}
% The \LaTeX{} options are represented by a comma-delimited list of
% \meta{\meta{key}=\meta{value}} pairs. For boolean options, the
% \meta{=\meta{value}} part is optional, and \meta{\meta{key}} will be
% interpreted as \meta{\meta{key}=true}.
%
% The \LaTeX{} options map directly to the options recognized by the plain
% \TeX{} interface (see Section \ref{sec:texoptions}) and to the markdown token
% renderers and their prototypes recognized by the plain \TeX{} interface (see
% Sections \ref{sec:texrenderersuser} and \ref{sec:texrendererprototypes}).
%
% The \LaTeX{} options may be specified when loading the \LaTeX{} package (see
% Section \ref{sec:latexinterface}), when using the \envm{markdown*} \LaTeX{}
% environment, or via the \mdef{markdownSetup} macro.  The \m{markdownSetup}
% macro receives the options to set up as its only argument.
%  \begin{macrocode}
\newcommand\markdownSetup[1]{%
  \setkeys{markdownOptions}{#1}}%
%    \end{macrocode}
%
% \paragraph{Plain \TeX{} Interface Options}
% The following options map directly to the option macros exposed by the plain
% \TeX{} interface (see Section \ref{sec:texoptions}).
%  \begin{macrocode}
\RequirePackage{keyval}
\define@key{markdownOptions}{helperScriptFileName}{%
  \def\markdownOptionHelperScriptFileName{#1}}%
\define@key{markdownOptions}{inputTempFileName}{%
  \def\markdownOptionInputTempFileName{#1}}%
\define@key{markdownOptions}{outputTempFileName}{%
  \def\markdownOptionOutputTempFileName{#1}}%
\define@key{markdownOptions}{blankBeforeBlockquote}[true]{%
  \def\markdownOptionBlankBeforeBlockquote{#1}}%
\define@key{markdownOptions}{blankBeforeHeading}[true]{%
  \def\markdownOptionBlankBeforeHeading{#1}}%
\define@key{markdownOptions}{cacheDir}{%
  \def\markdownOptionCacheDir{#1}}%
\define@key{markdownOptions}{definitionLists}[true]{%
  \def\markdownOptionDefinitionLists{#1}}%
\define@key{markdownOptions}{hashEnumerators}[true]{%
  \def\markdownOptionHashEnumerators{#1}}%
\define@key{markdownOptions}{hybrid}[true]{%
  \def\markdownOptionHybrid{#1}}%
\define@key{markdownOptions}{footnotes}[true]{%
  \def\markdownOptionFootnotes{#1}}%
\define@key{markdownOptions}{preserveTabs}[true]{%
  \def\markdownOptionPreserveTabs{#1}}%
\define@key{markdownOptions}{smartEllipses}[true]{%
  \def\markdownOptionSmartEllipses{#1}}%
\define@key{markdownOptions}{startNumber}[true]{%
  \def\markdownOptionStartNumber{#1}}%
%    \end{macrocode}
%
% The following example \LaTeX{} code showcases a possible configuration of
% plain \TeX{} interface options \m{markdownOptionHybrid},
% \m{markdownOptionSmartEllipses}, and \m{markdownOptionCacheDir}.
% \begin{Verbatim}
% \markdownSetup{
%   hybrid,
%   smartEllipses,
%   cacheDir = /tmp,
% }
% \end{Verbatim}
%
% \paragraph{Plain \TeX{} Markdown Token Renderers}
% The \LaTeX{} interface recognizes an option with the \t`renderers` key,
% whose value must be a list of options that map directly to the markdown token
% renderer macros exposed by the plain \TeX{} interface (see Section
% \ref{sec:texrenderersuser}).
%  \begin{macrocode}
\define@key{markdownOptions}{renderers}{%
  \setkeys{markdownRenderers}{#1}}%
\define@key{markdownRenderers}{lineBreak}{%
  \renewcommand\markdownRendererLineBreak{#1}}%
\define@key{markdownRenderers}{ellipsis}{%
  \renewcommand\markdownRendererEllipsis{#1}}%
\define@key{markdownRenderers}{codeSpan}{%
  \renewcommand\markdownRendererCodeSpan[1]{#1}}%
\define@key{markdownRenderers}{link}{%
  \renewcommand\markdownRendererLink[3]{#1}}%
\define@key{markdownRenderers}{image}{%
  \renewcommand\markdownRendererImage[3]{#1}}%
\define@key{markdownRenderers}{ulBegin}{%
  \renewcommand\markdownRendererUlBegin{#1}}%
\define@key{markdownRenderers}{ulBeginTight}{%
  \renewcommand\markdownRendererUlBeginTight{#1}}%
\define@key{markdownRenderers}{ulItem}{%
  \renewcommand\markdownRendererUlItem{#1}}%
\define@key{markdownRenderers}{ulEnd}{%
  \renewcommand\markdownRendererUlEnd{#1}}%
\define@key{markdownRenderers}{ulEndTight}{%
  \renewcommand\markdownRendererUlEndTight{#1}}%
\define@key{markdownRenderers}{olBegin}{%
  \renewcommand\markdownRendererOlBegin{#1}}%
\define@key{markdownRenderers}{olBeginTight}{%
  \renewcommand\markdownRendererOlBeginTight{#1}}%
\define@key{markdownRenderers}{olItem}{%
  \renewcommand\markdownRendererOlItem{#1}}%
\define@key{markdownRenderers}{olItemWithNumber}{%
  \renewcommand\markdownRendererOlItemWithNumber[1]{#1}}%
\define@key{markdownRenderers}{olEnd}{%
  \renewcommand\markdownRendererOlEnd{#1}}%
\define@key{markdownRenderers}{olEndTight}{%
  \renewcommand\markdownRendererOlEndTight{#1}}%
\define@key{markdownRenderers}{dlBegin}{%
  \renewcommand\markdownRendererDlBegin{#1}}%
\define@key{markdownRenderers}{dlBeginTight}{%
  \renewcommand\markdownRendererDlBeginTight{#1}}%
\define@key{markdownRenderers}{dlItem}{%
  \renewcommand\markdownRendererDlItem[1]{#1}}%
\define@key{markdownRenderers}{dlEnd}{%
  \renewcommand\markdownRendererDlEnd{#1}}%
\define@key{markdownRenderers}{dlEndTight}{%
  \renewcommand\markdownRendererDlEndTight{#1}}%
\define@key{markdownRenderers}{emphasis}{%
  \renewcommand\markdownRendererEmphasis[1]{#1}}%
\define@key{markdownRenderers}{strongEmphasis}{%
  \renewcommand\markdownRendererStrongEmphasis[1]{#1}}%
\define@key{markdownRenderers}{blockQuoteBegin}{%
  \renewcommand\markdownRendererBlockQuoteBegin{#1}}%
\define@key{markdownRenderers}{blockQuoteEnd}{%
  \renewcommand\markdownRendererBlockQuoteEnd{#1}}%
\define@key{markdownRenderers}{inputVerbatim}{%
  \renewcommand\markdownRendererInputVerbatim[1]{#1}}%
\define@key{markdownRenderers}{headingOne}{%
  \renewcommand\markdownRendererHeadingOne[1]{#1}}%
\define@key{markdownRenderers}{headingTwo}{%
  \renewcommand\markdownRendererHeadingTwo[1]{#1}}%
\define@key{markdownRenderers}{headingThree}{%
  \renewcommand\markdownRendererHeadingThree[1]{#1}}%
\define@key{markdownRenderers}{headingFour}{%
  \renewcommand\markdownRendererHeadingFour[1]{#1}}%
\define@key{markdownRenderers}{headingFive}{%
  \renewcommand\markdownRendererHeadingFive[1]{#1}}%
\define@key{markdownRenderers}{headingSix}{%
  \renewcommand\markdownRendererHeadingSix[1]{#1}}%
\define@key{markdownRenderers}{horizontalRule}{%
  \renewcommand\markdownRendererHorizontalRule{#1}}%
\define@key{markdownRenderers}{footnote}{%
  \renewcommand\markdownRendererFootnote[1]{#1}}%
%    \end{macrocode}
%
% The following example \LaTeX{} code showcases a possible configuration of the
% \m{markdownRendererLink} and \m{markdownRendererEmphasis} markdown token
% renderers.
% \begin{Verbatim}
% \markdownSetup{
%   renderers = {
%     link = {#3},                   % Render links as the link title.
%     emphasis = {\emph{#1}},    % Render emphasized text via `\emph`.
%   }
% }
% \end{Verbatim}
%
% \paragraph{Plain \TeX{} Markdown Token Renderer Prototypes}
% The \LaTeX{} interface recognizes an option with the \t`rendererPrototypes`
% key, whose value must be a list of options that map directly to the markdown
% token renderer prototype macros exposed by the plain \TeX{} interface (see
% Section \ref{sec:texrendererprototypes}).
%  \begin{macrocode}
\define@key{markdownOptions}{rendererPrototypes}{%
  \setkeys{markdownRendererPrototypes}{#1}}%
\define@key{markdownRendererPrototypes}{lineBreak}{%
  \renewcommand\markdownRendererLineBreakPrototype{#1}}%
\define@key{markdownRendererPrototypes}{ellipsis}{%
  \renewcommand\markdownRendererEllipsisPrototype{#1}}%
\define@key{markdownRendererPrototypes}{codeSpan}{%
  \renewcommand\markdownRendererCodeSpanPrototype[1]{#1}}%
\define@key{markdownRendererPrototypes}{link}{%
  \renewcommand\markdownRendererLink[3]{#1}}%
\define@key{markdownRendererPrototypes}{image}{%
  \renewcommand\markdownRendererImage[3]{#1}}%
\define@key{markdownRendererPrototypes}{ulBegin}{%
  \renewcommand\markdownRendererUlBeginPrototype{#1}}%
\define@key{markdownRendererPrototypes}{ulBeginTight}{%
  \renewcommand\markdownRendererUlBeginTightPrototype{#1}}%
\define@key{markdownRendererPrototypes}{ulItem}{%
  \renewcommand\markdownRendererUlItemPrototype{#1}}%
\define@key{markdownRendererPrototypes}{ulEnd}{%
  \renewcommand\markdownRendererUlEndPrototype{#1}}%
\define@key{markdownRendererPrototypes}{ulEndTight}{%
  \renewcommand\markdownRendererUlEndTightPrototype{#1}}%
\define@key{markdownRendererPrototypes}{olBegin}{%
  \renewcommand\markdownRendererOlBeginPrototype{#1}}%
\define@key{markdownRendererPrototypes}{olBeginTight}{%
  \renewcommand\markdownRendererOlBeginTightPrototype{#1}}%
\define@key{markdownRendererPrototypes}{olItem}{%
  \renewcommand\markdownRendererOlItemPrototype{#1}}%
\define@key{markdownRendererPrototypes}{olItemWithNumber}{%
  \renewcommand\markdownRendererOlItemWithNumberPrototype[1]{#1}}%
\define@key{markdownRendererPrototypes}{olEnd}{%
  \renewcommand\markdownRendererOlEndPrototype{#1}}%
\define@key{markdownRendererPrototypes}{olEndTight}{%
  \renewcommand\markdownRendererOlEndTightPrototype{#1}}%
\define@key{markdownRendererPrototypes}{dlBegin}{%
  \renewcommand\markdownRendererDlBeginPrototype{#1}}%
\define@key{markdownRendererPrototypes}{dlBeginTight}{%
  \renewcommand\markdownRendererDlBeginTightPrototype{#1}}%
\define@key{markdownRendererPrototypes}{dlItem}{%
  \renewcommand\markdownRendererDlItemPrototype[1]{#1}}%
\define@key{markdownRendererPrototypes}{dlEnd}{%
  \renewcommand\markdownRendererDlEndPrototype{#1}}%
\define@key{markdownRendererPrototypes}{dlEndTight}{%
  \renewcommand\markdownRendererDlEndTightPrototype{#1}}%
\define@key{markdownRendererPrototypes}{emphasis}{%
  \renewcommand\markdownRendererEmphasisPrototype[1]{#1}}%
\define@key{markdownRendererPrototypes}{strongEmphasis}{%
  \renewcommand\markdownRendererStrongEmphasisPrototype[1]{#1}}%
\define@key{markdownRendererPrototypes}{blockQuoteBegin}{%
  \renewcommand\markdownRendererBlockQuoteBeginPrototype{#1}}%
\define@key{markdownRendererPrototypes}{blockQuoteEnd}{%
  \renewcommand\markdownRendererBlockQuoteEndPrototype{#1}}%
\define@key{markdownRendererPrototypes}{inputVerbatim}{%
  \renewcommand\markdownRendererInputVerbatimPrototype[1]{#1}}%
\define@key{markdownRendererPrototypes}{headingOne}{%
  \renewcommand\markdownRendererHeadingOnePrototype[1]{#1}}%
\define@key{markdownRendererPrototypes}{headingTwo}{%
  \renewcommand\markdownRendererHeadingTwoPrototype[1]{#1}}%
\define@key{markdownRendererPrototypes}{headingThree}{%
  \renewcommand\markdownRendererHeadingThreePrototype[1]{#1}}%
\define@key{markdownRendererPrototypes}{headingFour}{%
  \renewcommand\markdownRendererHeadingFourPrototype[1]{#1}}%
\define@key{markdownRendererPrototypes}{headingFive}{%
  \renewcommand\markdownRendererHeadingFivePrototype[1]{#1}}%
\define@key{markdownRendererPrototypes}{headingSix}{%
  \renewcommand\markdownRendererHeadingSixPrototype[1]{#1}}%
\define@key{markdownRendererPrototypes}{horizontalRule}{%
  \renewcommand\markdownRendererHorizontalRulePrototype{#1}}%
\define@key{markdownRendererPrototypes}{footnote}{%
  \renewcommand\markdownRendererFootnotePrototype[1]{#1}}%
%    \end{macrocode}
%
% The following example \LaTeX{} code showcases a possible configuration of the
% \m{markdownRendererImagePrototype} and \m{markdownRendererCodeSpanPrototype}
% markdown token renderer prototypes.
% \begin{Verbatim}
% \markdownSetup{
%   renderers = {
%     image = {\includegraphics{#2}},
%     codeSpan = {\texttt{#1}},    % Render inline code via `\texttt`.
%   }
% }
% \end{Verbatim}
%
% \iffalse
%</latex>
%<*context>
% \fi\subsection{\Hologo{ConTeXt} Interface}\label{sec:contextinterface}
% The \Hologo{ConTeXt} interface provides a start-stop macro pair for the
% typesetting of markdown input from within \Hologo{ConTeXt}. The rest of the
% interface is inherited from the plain \TeX{} interface (see Section
% \ref{sec:texinterface}).
%  \begin{macrocode}
\writestatus{loading}{ConTeXt User Module / markdown}%
\unprotect
%    \end{macrocode}
%
% The \Hologo{ConTeXt} interface is implemented by the
% \t`t-markdown.tex` \Hologo{ConTeXt} module file that can be loaded as follows:
% \begin{Verbatim}
% \usemodule[t][markdown]
% \end{Verbatim}
% It is expected that the special plain \TeX{} characters have the expected
% category codes, when \m{input}ting the file.
%
% \subsubsection{Typesetting Markdown}
% The interface exposes the \mdef{startmarkdown} and \mdef{stopmarkdown} macro
% pair for the typesetting of a markdown document fragment.
%  \begin{macrocode}
\let\startmarkdown\relax
\let\stopmarkdown\relax
%    \end{macrocode}
% You may prepend your own code to the \m{startmarkdown} macro and redefine the
% \m{stopmarkdown} macro to produce special effects before and after the
% markdown block.
%
% Note that the \m{startmarkdown} and \m{stopmarkdown} macros
% are subject to the same limitations as the \m{markdownBegin} and
% \m{markdownEnd} macros exposed by the plain \TeX{} interface.
%
% The following example \Hologo{ConTeXt} code showcases the usage of the
% \m{startmarkdown} and \m{stopmarkdown} macros:
% \begin{Verbatim}
% \usemodule[t][markdown]
% \starttext
% \startmarkdown
% _Hello_ **world** ...
% \stopmarkdown
% \stoptext
% \end{Verbatim}
% 
% \section{Technical Documentation}\label{sec:implementation}
% This part of the manual describes the implementation of the interfaces
% exposed by the package (see Section \ref{sec:interfaces}) and is aimed at the
% developers of the package, as well as the curious users.
%
% \iffalse
%</context>
%<*lua>
% \fi\subsection{Lua Implementation}\label{sec:luaimplementation}
% The Lua implementation implements \luamdef{writer} and \luamdef{reader}
% objects that provide the conversion from markdown to plain \TeX{}.
%
% The Lunamark Lua module implements writers for the conversion to various
% other formats, such as DocBook, Groff, or \acro{HTML}. These were stripped
% from the module and the remaining markdown reader and plain \TeX{} writer
% were hidden behind the converter functions exposed by the Lua interface (see
% Section \ref{sec:luainterface}).
%
%  \begin{macrocode}
local upper, gsub, format, length =
  string.upper, string.gsub, string.format, string.len
local concat = table.concat
local P, R, S, V, C, Cg, Cb, Cmt, Cc, Ct, B, Cs, any =
  lpeg.P, lpeg.R, lpeg.S, lpeg.V, lpeg.C, lpeg.Cg, lpeg.Cb,
  lpeg.Cmt, lpeg.Cc, lpeg.Ct, lpeg.B, lpeg.Cs, lpeg.P(1)
%    \end{macrocode}
% 
% \subsubsection{Utility Functions}
% This section documents the utility functions used by the Lua code. These
% functions are encapsulated in the \t`util` object. The functions were
% originally located in the \t`lunamark/util.lua` file in the Lunamark Lua
% module.
%  \begin{macrocode}
local util = {}
%    \end{macrocode}
% 
% The \luamdef{util.err} method prints an error message \t`msg` and exits.
% If \t`exit_code` is provided, it specifies the exit code.  Otherwise, the
% exit code will be 1.
%  \begin{macrocode}
function util.err(msg, exit_code)
  io.stderr:write("markdown.lua: " .. msg .. "\n")
  os.exit(exit_code or 1)
end
%    \end{macrocode}
%
% The \luamdef{util.cache} method computes the digest of \t`string` and
% \t`salt`, adds the \t`suffix` and looks into the directory \t`dir`, whether a
% file with such a name exists. If it does not, it gets created with
% \t`transform(string)` as its content. The filename is then returned.
%  \begin{macrocode}
function util.cache(dir, string, salt, transform, suffix)
  local digest = md5.sumhexa(string .. (salt or ""))
  local name = util.pathname(dir, digest .. suffix)
  local file = io.open(name, "r")
  if file == nil then -- If no cache entry exists, then create a new one.
    local file = assert(io.open(name, "w"))
    local result = string
    if transform ~= nil then
      result = transform(result)
    end
    assert(file:write(result))
    assert(file:close())
  end
  return name
end
%    \end{macrocode}
%
% The \luamdef{util.table_copy} method creates a shallow copy of a table \t`t`
% and its metatable.
%  \begin{macrocode}
function util.table_copy(t)
  local u = { }
  for k, v in pairs(t) do u[k] = v end
  return setmetatable(u, getmetatable(t))
end
%    \end{macrocode}
%
% The \luamdef{util.expand_tabs_in_line} expands tabs in string \t`s`. If
% \t`tabstop` is specified, it is used as the tab stop width. Otherwise,
% the tab stop width of 4 characters is used. The method is a copy of the tab
% expansion algorithm from \cite[Chapter~21]{ierusalimschy13}.
%  \begin{macrocode}
function util.expand_tabs_in_line(s, tabstop)
  local tab = tabstop or 4
  local corr = 0
  return (s:gsub("()\t", function(p)
            local sp = tab - (p - 1 + corr) % tab
            corr = corr - 1 + sp
            return string.rep(" ", sp)
          end))
end
%    \end{macrocode}
%
% The \luamdef{util.walk} method walks a rope \t`t`, applying a function \t`f`
% to each leaf element in order. A rope is an array whose elements may be
% ropes, strings, numbers, or functions.  If a leaf element is a function, call
% it and get the return value before proceeding.
%  \begin{macrocode}
function util.walk(t, f)
  local typ = type(t)
  if typ == "string" then
    f(t)
  elseif typ == "table" then
    local i = 1
    local n
    n = t[i]
    while n do
      util.walk(n, f)
      i = i + 1
      n = t[i]
    end
  elseif typ == "function" then
    local ok, val = pcall(t)
    if ok then
      util.walk(val,f)
    end
  else
    f(tostring(t))
  end
end
%    \end{macrocode}
%
% The \luamdef{util.flatten} method flattens an array \t`ary` that does not
% contain cycles and returns the result.
%  \begin{macrocode}
function util.flatten(ary)
  local new = {}
  for _,v in ipairs(ary) do
    if type(v) == "table" then
      for _,w in ipairs(util.flatten(v)) do
        new[#new + 1] = w
      end
    else
      new[#new + 1] = v
    end
  end
  return new
end
%    \end{macrocode}
%
% The \luamdef{util.rope_to_string} method converts a rope \t`rope` to a
% string and returns it. For the definition of a rope, see the definition of
% the \luam{util.walk} method.
%  \begin{macrocode}
function util.rope_to_string(rope)
  local buffer = {}
  util.walk(rope, function(x) buffer[#buffer + 1] = x end)
  return table.concat(buffer)
end
%    \end{macrocode}
%
% The \luamdef{util.rope_last} method retrieves the last item in a rope. For
% the definition of a rope, see the definition of the \luam{util.walk} method.
%  \begin{macrocode}
function util.rope_last(rope)
  if #rope == 0 then
    return nil
  else
    local l = rope[#rope]
    if type(l) == "table" then
      return util.rope_last(l)
    else
      return l
    end
  end
end
%    \end{macrocode}
%
% Given an array \t`ary` and a string \t`x`, the \luamdef{util.intersperse}
% method returns an array \t`new`, such that \t`ary[i] == new[2*(i-1)+1]` and
% \t`new[2*i] == x` for all $1\leq\t`i`\leq\t`\#ary`$.
%  \begin{macrocode}
function util.intersperse(ary, x)
  local new = {}
  local l = #ary
  for i,v in ipairs(ary) do
    local n = #new
    new[n + 1] = v
    if i ~= l then
      new[n + 2] = x
    end
  end
  return new
end
%    \end{macrocode}
% 
% Given an array \t`ary` and a function \t`f`, the \luamdef{util.map} method
% returns an array \t`new`, such that \t`new[i] == f(ary[i])` for all
% $1\leq\t`i`\leq\t`\#ary`$.
%  \begin{macrocode}
function util.map(ary, f)
  local new = {}
  for i,v in ipairs(ary) do
    new[i] = f(v)
  end
  return new
end
%    \end{macrocode}
%
% Given a table \t`char_escapes` mapping escapable characters to escaped
% strings and optionally a table \t`string_escapes` mapping escapable strings
% to escaped strings, the \luamdef{util.escaper} method returns an escaper
% function that escapes all occurances of escapable strings and characters (in
% this order).
% 
% The method uses \pkg{LPeg}, which is faster than the Lua \t`string.gsub`
% built-in method.
%  \begin{macrocode}
function util.escaper(char_escapes, string_escapes)
%    \end{macrocode}
% Build a string of escapable characters.
%  \begin{macrocode}
  local char_escapes_list = ""
  for i,_ in pairs(char_escapes) do
    char_escapes_list = char_escapes_list .. i
  end
%    \end{macrocode}
% Create an \pkg{LPeg} capture \t`escapable` that produces the escaped string
% corresponding to the matched escapable character.
%  \begin{macrocode}
  local escapable = S(char_escapes_list) / char_escapes
%    \end{macrocode}
% If \t`string_escapes` is provided, turn \t`escapable` into the
% {\catcode`\_=8\[
%   \sum_{(\t`k`,\t`v`)\in\t`string\_escapes`}\t`P(k) / v` + \t`escapable`
% \]}^^A
% capture that replaces any occurance of the string \t`k` with the string
% \t`v` for each $(\t`k`, \t`v`)\in\t`string\_escapes`$. Note that the pattern
% summation is not commutative and the its operands are inspected in the
% summation order during the matching. As a corrolary, the strings always
% take precedence over the characters.
%  \begin{macrocode}
  if string_escapes then
    for k,v in pairs(string_escapes) do
      escapable = P(k) / v + escapable
    end
  end
%    \end{macrocode}
% Create an \pkg{LPeg} capture \t`escape_string` that captures anything
% \t`escapable` does and matches any other unmatched characters.
%  \begin{macrocode}
  local escape_string = Cs((escapable + any)^0)
%    \end{macrocode}
% Return a function that matches the input string \t`s` against the
% \t`escape_string` capture.
%  \begin{macrocode}
  return function(s)
    return lpeg.match(escape_string, s)
  end
end
%    \end{macrocode}
%
% The \luamdef{util.pathname} method produces a pathname out of a directory
% name \t`dir` and a filename \t`file` and returns it.
%  \begin{macrocode}
function util.pathname(dir, file)
  if #dir == 0 then
    return file
  else
    return dir .. "/" .. file
  end
end
%    \end{macrocode}
% \subsubsection{Plain \TeX{} Writer}\label{sec:texwriter}
% This section documents the \luam{writer} object, which implements the
% routines for producing the \TeX{} output. The object is an amalgamate of the
% generic, \TeX{}, \LaTeX{} writer objects that were located in the
% \t`lunamark/writer/generic.lua`, \t`lunamark/writer/tex.lua`, and
% \t`lunamark/writer/latex.lua` files in the Lunamark Lua module.
%
% Although not specified in the Lua interface (see Section
% \ref{sec:luainterface}), the \luam{writer} object is exported, so that the
% curious user could easily tinker with the methods of the objects produced by
% the \luam{writer.new} method described below. The user should be aware,
% however, that the implementation may change in a future revision.
%  \begin{macrocode}
M.writer = {}
%    \end{macrocode}
%
% The \luamdef{writer.new} method creates and returns a new \TeX{} writer
% object associated with the Lua interface options (see Section
% \ref{sec:luaoptions}) \t`options`. When \t`options` are unspecified, it is
% assumed that an empty table was passed to the method.
%
% The objects produced by the \luam{writer.new} method expose instance methods
% and variables of their own. As a convention, I will refer to these
% \meta{member}s as \t`writer->`\meta{member}.
%  \begin{macrocode}
function M.writer.new(options)
  local self = {}
  options = options or {}
%    \end{macrocode}
%
% Make the \t`options` table inherit from the \luam{defaultOptions} table.
%  \begin{macrocode}
  setmetatable(options, { __index = function (_, key)
    return defaultOptions[key] end })
%    \end{macrocode}
%
% Define \luamdef{writer->suffix} as the suffix of the produced cache files.
%  \begin{macrocode}
  self.suffix = ".tex"
%    \end{macrocode}
%
% Define \luamdef{writer->space} as the output format of a space character.
%  \begin{macrocode}
  self.space = " "
%    \end{macrocode}
%
% Define \luamdef{writer->plain} as a function that will transform an input
% plain text block \t`s` to the output format.
%  \begin{macrocode}
  function self.plain(s)
    return s
  end
%    \end{macrocode}
%
% Define \luamdef{writer->paragraph} as a function that will transform an
% input paragraph \t`s` to the output format.
%  \begin{macrocode}
  function self.paragraph(s)
    return s
  end
%    \end{macrocode}
%
% Define \luamdef{writer->pack} as a function that will take the filename
% \t`name` of the output file prepared by the reader and transform it to the
% output format.
%  \begin{macrocode}
  function self.pack(name)
    return [[\input"]] .. name .. [["\relax]]
  end
%    \end{macrocode}
%
% Define \luamdef{writer->interblocksep} as the output format of a block
% element separator.
%  \begin{macrocode}
  self.interblocksep = "\n\n"
%    \end{macrocode}
%
% Define \luamdef{writer->containersep} as the output format of a container
% separator.
%  \begin{macrocode}
  self.containersep = "\n\n"
%    \end{macrocode}
%
% Define \luamdef{writer->eof} as the end of file marker in the output format.
%  \begin{macrocode}
  self.eof = [[\relax]]
%    \end{macrocode}
%
% Define \luamdef{writer->linebreak} as the output format of a forced line break.
%  \begin{macrocode}
  self.linebreak = "\\markdownRendererLineBreak "
%    \end{macrocode}
%
% Define \luamdef{writer->ellipsis} as the output format of an ellipsis.
%  \begin{macrocode}
  self.ellipsis = "\\markdownRendererEllipsis{}"
%    \end{macrocode}
%
% Define \luamdef{writer->hrule} as the output format of a horizontal rule.
%  \begin{macrocode}
  self.hrule = "\\markdownRendererHorizontalRule "
%    \end{macrocode}
%
% Define a table \luamdef{escaped} containing the mapping from special plain
% \TeX{} characters to their escaped variants.
%  \begin{macrocode}
  local escaped = {
     ["{"] = "\\{",
     ["}"] = "\\}",
     ["$"] = "\\$",
     ["%"] = "\\%",
     ["&"] = "\\&",
     ["_"] = "\\_",
     ["#"] = "\\#",
     ["^"] = "\\^{}",
     ["\\"] = "\\char92{}",
     ["~"] = "\\char126{}",
     ["|"] = "\\char124{}", }
%    \end{macrocode}
% Use the \luam{escaped} table to create an escaper function \luamdef{escape}.
%  \begin{macrocode}
  local escape = util.escaper(escaped)
%    \end{macrocode}
%
% Define \luamdef{writer->string} as a function that will transform an input
% plain text span \t`s` to the output format. If the \Opt{hybrid} option is
% \t`true`, use an identity function. Otherwise, use the \luam{escape}
% function.
%  \begin{macrocode}
  if options.hybrid then
    self.string = function(s) return s end
  else
    self.string = escape
  end
%    \end{macrocode}
%
% Define \luamdef{writer->code} as a function that will transform an input
% inlined code span \t`s` to the output format.
%  \begin{macrocode}
  function self.code(s)
    return {"\\markdownRendererCodeSpan{",escape(s),"}"}
  end
%    \end{macrocode}
%
% Define \luamdef{writer->link} as a function that will transform an input
% hyperlink to the output format, where \t`lab` corresponds to thelabel,
% \t`src` to \acro{uri}, and \t`tit` to the title of the link.
%  \begin{macrocode}
  function self.link(lab,src,tit)
    return {"\\markdownRendererLink{",lab,"}",
                          "{",self.string(src),"}",
                          "{",self.string(tit),"}"}
  end
%    \end{macrocode}
%
% Define \luamdef{writer->image} as a function that will transform an input
% image to the output format, where \t`lab` corresponds to the label, \t`src`
% to the \acro{url}, and \t`tit` to the title of the image.
%  \begin{macrocode}
  function self.image(lab,src,tit)
    return {"\\markdownRendererImage{",lab,"}",
                           "{",self.string(src),"}",
                           "{",self.string(tit),"}"}
  end
%    \end{macrocode}
%
% Define \luamdef{writer->bulletlist} as a function that will transform an input
% bulleted list to the output format, where \t`items` is an array of the list
% items and \t`tight` specifies, whether the list is tight or not.
%  \begin{macrocode}
  local function ulitem(s)
    return {"\\markdownRendererUlItem ",s}
  end

  function self.bulletlist(items,tight)
    local buffer = {}
    for _,item in ipairs(items) do
      buffer[#buffer + 1] = ulitem(item)
    end
    local contents = util.intersperse(buffer,"\n")
    if tight then
      return {"\\markdownRendererUlBeginTight\n",contents,
        "\n\\markdownRendererUlEndTight "}
    else
      return {"\\markdownRendererUlBegin\n",contents,
        "\n\\markdownRendererUlEnd "}
    end
  end
%    \end{macrocode}
%
% Define \luamdef{writer->ollist} as a function that will transform an input
% ordered list to the output format, where \t`items` is an array of the list
% items and \t`tight` specifies, whether the list is tight or not. If the
% optional parameter \t`startnum` is present, it should be used as the number
% of the first list item.
%  \begin{macrocode}
  local function olitem(s,num)
    if num ~= nil then
      return {"\\markdownRendererOlItemWithNumber{",num,"}",s}
    else
      return {"\\markdownRendererOlItem ",s}
    end
  end

  function self.orderedlist(items,tight,startnum)
    local buffer = {}
    local num = startnum
    for _,item in ipairs(items) do
      buffer[#buffer + 1] = olitem(item,num)
      if num ~= nil then
        num = num + 1
      end
    end
    local contents = util.intersperse(buffer,"\n")
    if tight then
      return {"\\markdownRendererOlBeginTight\n",contents,
        "\n\\markdownRendererOlEndTight "}
    else
      return {"\\markdownRendererOlBegin\n",contents,
        "\n\\markdownRendererOlEnd "}
    end
  end
%    \end{macrocode}
%
% Define \luamdef{writer->definitionlist} as a function that will transform an
% input definition list to the output format, where \t`items` is an array of
% tables, each of the form \t`\{ term = t, definitions = defs \}`, where \t`t`
% is a term and \t`defs` is an array of definitions. \t`tight` specifies,
% whether the list is tight or not.
%  \begin{macrocode}
  local function dlitem(term,defs)
      return {"\\markdownRendererDlItem{",term,"}\n",defs}
  end

  function self.definitionlist(items,tight)
    local buffer = {}
    for _,item in ipairs(items) do
      buffer[#buffer + 1] = dlitem(item.term,
        util.intersperse(item.definitions, self.interblocksep))
    end
    local contents = util.intersperse(buffer, self.containersep)
    if tight then
      return {"\\markdownRendererDlBeginTight\n\n", contents,
        "\n\n\\markdownRendererDlEndTight\n"}
    else
      return {"\\markdownRendererDlBegin\n\n", contents,
        "\n\n\\markdownRendererDlEnd\n"}
    end
  end
%    \end{macrocode}
%
% Define \luamdef{writer->emphasis} as a function that will transform an
% emphasized span \t`s` of input text to the output format.
%  \begin{macrocode}
  function self.emphasis(s)
    return {"\\markdownRendererEmphasis{",s,"}"}
  end
%    \end{macrocode}
%
% Define \luamdef{writer->strong} as a function that will transform a strongly
% emphasized span \t`s` of input text to the output format.
%  \begin{macrocode}
  function self.strong(s)
    return {"\\markdownRendererStrongEmphasis{",s,"}"}
  end
%    \end{macrocode}
%
% Define \luamdef{writer->blockquote} as a function that will transform an
% input block quote \t`s` to the output format.
%  \begin{macrocode}
  function self.blockquote(s)
    return {"\\markdownRendererBlockQuoteBegin\n",s,
      "\n\\markdownRendererBlockQuoteEnd "}
  end
%    \end{macrocode}
%
% Define \luamdef{writer->verbatim} as a function that will transform an
% input code block \t`s` to the output format.
%  \begin{macrocode}
  function self.verbatim(s)
    local name = util.cache(options.cacheDir, s, nil, nil, ".verbatim")
    return {"\\markdownRendererInputVerbatim{",name,"}"}
  end
%    \end{macrocode}
%
% Define \luamdef{writer->heading} as a function that will transform an
% input heading \t`s` at level \t`level` to the output format.
%  \begin{macrocode}
  function self.heading(s,level)
    local cmd
    if level == 1 then
      cmd = "\\markdownRendererHeadingOne"
    elseif level == 2 then
      cmd = "\\markdownRendererHeadingTwo"
    elseif level == 3 then
      cmd = "\\markdownRendererHeadingThree"
    elseif level == 4 then
      cmd = "\\markdownRendererHeadingFour"
    elseif level == 5 then
      cmd = "\\markdownRendererHeadingFive"
    elseif level == 6 then
      cmd = "\\markdownRendererHeadingSix"
    else
      cmd = ""
    end
    return {cmd,"{",s,"}"}
  end
%    \end{macrocode}
%
% Define \luamdef{writer->footnote} as a function that will transform an
% input footnote \t`s` to the output format.
%  \begin{macrocode}
  function self.note(s)
    return {"\\markdownRendererFootnote{",s,"}"}
  end

  return self
end
%    \end{macrocode}
% 
% \subsubsection{Markdown Reader}\label{sec:markdownreader}
% This section documents the \luam{reader} object, which implements the
% routines for parsing the markdown input. The object corresponds to the
% markdown reader object that was located in the
% \t`lunamark/reader/markdown.lua` file in the Lunamark Lua module.
%
% Although not specified in the Lua interface (see Section
% \ref{sec:luainterface}), the \luam{reader} object is exported, so that the
% curious user could easily tinker with the methods of the objects produced by
% the \luam{reader.new} method described below. The user should be aware,
% however, that the implementation may change in a future revision.
%
% The \luamdef{reader.new} method creates and returns a new \TeX{} reader
% object associated with the Lua interface options (see Section
% \ref{sec:luaoptions}) \t`options` and with a writer object \t`writer`. When
% \t`options` are unspecified, it is assumed that an empty table was passed to
% the method.
%
% The objects produced by the \luam{reader.new} method expose instance methods
% and variables of their own. As a convention, I will refer to these
% \meta{member}s as \t`reader->`\meta{member}.
%  \begin{macrocode}
M.reader = {}
function M.reader.new(writer, options)
  local self = {}
  options = options or {}
%    \end{macrocode}
%
% Make the \t`options` table inherit from the \luam{defaultOptions} table.
%  \begin{macrocode}
  setmetatable(options, { __index = function (_, key)
    return defaultOptions[key] end })
%    \end{macrocode}
%
% \paragraph{Top Level Helper Functions}
% Define \luamdef{normalize_tag} as a function that normalizes a markdown
% reference tag by lowercasing it, and by collapsing any adjacent whitespace
% characters.
%  \begin{macrocode}
  local function normalize_tag(tag)
    return unicode.utf8.lower(
      gsub(util.rope_to_string(tag), "[ \n\r\t]+", " "))
  end
%    \end{macrocode}
% 
% Define \luamdef{expandtabs} either as an identity function, when the
% \Opt{preserveTabs} Lua inrerface option is \t`true`, or to a function that
% expands tabs into spaces otherwise.
%  \begin{macrocode}
  local expandtabs
  if options.preserveTabs then
    expandtabs = function(s) return s end
  else
    expandtabs = function(s)
                   if s:find("\t") then
                     return s:gsub("[^\n]*", util.expand_tabs_in_line)
                   else
                     return s
                   end
                 end
  end
%    \end{macrocode}
% \paragraph{Top Level Parsing Functions}
%  \begin{macrocode}
  local syntax
  local blocks
  local inlines

  local parse_blocks =
    function(str)
      local res = lpeg.match(blocks, str)
      if res == nil then
        error(format("parse_blocks failed on:\n%s", str:sub(1,20)))
      else
        return res
      end
    end

  local parse_inlines =
    function(str)
      local res = lpeg.match(inlines, str)
      if res == nil then
        error(format("parse_inlines failed on:\n%s",
        str:sub(1,20)))
      else
        return res
      end
    end

  local parse_inlines_no_link =
    function(str)
      local res = lpeg.match(inlines_no_link, str)
      if res == nil then
        error(format("parse_inlines_no_link failed on:\n%s",
        str:sub(1,20)))
      else
        return res
      end
    end
%    \end{macrocode}
% \paragraph{Generic \acro{peg} Patterns}
%  \begin{macrocode}
  local percent                = P("%")
  local asterisk               = P("*")
  local dash                   = P("-")
  local plus                   = P("+")
  local underscore             = P("_")
  local period                 = P(".")
  local hash                   = P("#")
  local ampersand              = P("&")
  local backtick               = P("`")
  local less                   = P("<")
  local more                   = P(">")
  local space                  = P(" ")
  local squote                 = P("'")
  local dquote                 = P('"')
  local lparent                = P("(")
  local rparent                = P(")")
  local lbracket               = P("[")
  local rbracket               = P("]")
  local circumflex             = P("^")
  local slash                  = P("/")
  local equal                  = P("=")
  local colon                  = P(":")
  local semicolon              = P(";")
  local exclamation            = P("!")

  local digit                  = R("09")
  local hexdigit               = R("09","af","AF")
  local letter                 = R("AZ","az")
  local alphanumeric           = R("AZ","az","09")
  local keyword                = letter * alphanumeric^0

  local doubleasterisks        = P("**")
  local doubleunderscores      = P("__")
  local fourspaces             = P("    ")

  local any                    = P(1)
  local fail                   = any - 1
  local always                 = P("")

  local escapable              = S("\\`*_{}[]()+_.!#-~:^")
  local anyescaped             = P("\\") / "" * escapable
                               + any

  local tab                    = P("\t")
  local spacechar              = S("\t ")
  local spacing                = S(" \n\r\t")
  local newline                = P("\n")
  local nonspacechar           = any - spacing
  local tightblocksep          = P("\001")

  local specialchar
  if options.smartEllipses then
    specialchar                = S("*_`&[]!\\.")
  else
    specialchar                = S("*_`&[]!\\")
  end

  local normalchar             = any -
                                 (specialchar + spacing + tightblocksep)
  local optionalspace          = spacechar^0
  local spaces                 = spacechar^1
  local eof                    = - any
  local nonindentspace         = space^-3 * - spacechar
  local indent                 = space^-3 * tab
                               + fourspaces / ""
  local linechar               = P(1 - newline)

  local blankline              = optionalspace * newline / "\n"
  local blanklines             = blankline^0
  local skipblanklines         = (optionalspace * newline)^0
  local indentedline           = indent    /"" * C(linechar^1 * newline^-1)
  local optionallyindentedline = indent^-1 /"" * C(linechar^1 * newline^-1)
  local sp                     = spacing^0
  local spnl                   = optionalspace * (newline * optionalspace)^-1
  local line                   = linechar^0 * newline
                               + linechar^1 * eof
  local nonemptyline           = line - blankline

  local chunk = line * (optionallyindentedline - blankline)^0

  -- block followed by 0 or more optionally
  -- indented blocks with first line indented.
  local function indented_blocks(bl)
    return Cs( bl
             * (blankline^1 * indent * -blankline * bl)^0
             * blankline^1 )
  end
%    \end{macrocode}
% \paragraph{List \acro{peg} Patterns}
%  \begin{macrocode}
  local bulletchar = C(plus + asterisk + dash)

  local bullet     = ( bulletchar * #spacing * (tab + space^-3)
                     + space * bulletchar * #spacing * (tab + space^-2)
                     + space * space * bulletchar * #spacing * (tab + space^-1)
                     + space * space * space * bulletchar * #spacing
                     ) * -bulletchar

  if options.hashEnumerators then
    dig = digit + hash
  else
    dig = digit
  end

  local enumerator = C(dig^3 * period) * #spacing
                   + C(dig^2 * period) * #spacing * (tab + space^1)
                   + C(dig * period) * #spacing * (tab + space^-2)
                   + space * C(dig^2 * period) * #spacing
                   + space * C(dig * period) * #spacing * (tab + space^-1)
                   + space * space * C(dig^1 * period) * #spacing
%    \end{macrocode}
% \paragraph{Code Span \acro{peg} Patterns}
%  \begin{macrocode}
  local openticks   = Cg(backtick^1, "ticks")

  local function captures_equal_length(s,i,a,b)
    return #a == #b and i
  end

  local closeticks  = space^-1 *
                      Cmt(C(backtick^1) * Cb("ticks"), captures_equal_length)

  local intickschar = (any - S(" \n\r`"))
                    + (newline * -blankline)
                    + (space - closeticks)
                    + (backtick^1 - closeticks)

  local inticks     = openticks * space^-1 * C(intickschar^1) * closeticks
%    \end{macrocode}
% \paragraph{Tag \acro{peg} Patterns}
%  \begin{macrocode}
  local leader        = space^-3

  -- in balanced brackets, parentheses, quotes:
  local bracketed     = P{ lbracket
                         * ((anyescaped - (lbracket + rbracket
                             + blankline^2)) + V(1))^0
                         * rbracket }

  local inparens      = P{ lparent
                         * ((anyescaped - (lparent + rparent
                             + blankline^2)) + V(1))^0
                         * rparent }

  local squoted       = P{ squote * alphanumeric
                         * ((anyescaped - (squote + blankline^2))
                             + V(1))^0
                         * squote }

  local dquoted       = P{ dquote * alphanumeric
                         * ((anyescaped - (dquote + blankline^2))
                             + V(1))^0
                         * dquote }

  -- bracketed 'tag' for markdown links, allowing nested brackets:
  local tag           = lbracket
                      * Cs((alphanumeric^1
                           + bracketed
                           + inticks
                           + (anyescaped - (rbracket + blankline^2)))^0)
                      * rbracket

  -- url for markdown links, allowing balanced parentheses:
  local url           = less * Cs((anyescaped-more)^0) * more
                      + Cs((inparens + (anyescaped-spacing-rparent))^1)

  -- quoted text possibly with nested quotes:
  local title_s       = squote * Cs(((anyescaped-squote) + squoted)^0) *
                        squote

  local title_d       = dquote * Cs(((anyescaped-dquote) + dquoted)^0) *
                        dquote

  local title_p       = lparent
                      * Cs((inparens + (anyescaped-rparent))^0)
                      * rparent

  local title         = title_d + title_s + title_p

  local optionaltitle = spnl * title * spacechar^0
                      + Cc("")
%    \end{macrocode}
% \paragraph{Footnote \acro{peg} Patterns}
%  \begin{macrocode}
  local rawnotes = {}

  local function strip_first_char(s)
    return s:sub(2)
  end

  -- like indirect_link
  local function lookup_note(ref)
    return function()
      local found = rawnotes[normalize_tag(ref)]
      if found then
        return writer.note(parse_blocks(found))
      else
        return {"[^", ref, "]"}
      end
    end
  end

  local function register_note(ref,rawnote)
    rawnotes[normalize_tag(ref)] = rawnote
    return ""
  end

  local RawNoteRef = #(lbracket * circumflex) * tag / strip_first_char

  local NoteRef    = RawNoteRef / lookup_note

  local NoteBlock

  if options.footnotes then
    NoteBlock = leader * RawNoteRef * colon * spnl *
                indented_blocks(chunk) / register_note
  else
    NoteBlock = fail
  end
%    \end{macrocode}
% \paragraph{Link and Image \acro{peg} Patterns}
%  \begin{macrocode}
  -- List of references defined in the document
  local references

  -- add a reference to the list
  local function register_link(tag,url,title)
      references[normalize_tag(tag)] = { url = url, title = title }
      return ""
  end

  -- parse a reference definition:  [foo]: /bar "title"
  local define_reference_parser =
    leader * tag * colon * spacechar^0 * url * optionaltitle * blankline^1

  -- lookup link reference and return either
  -- the link or nil and fallback text.
  local function lookup_reference(label,sps,tag)
      local tagpart
      if not tag then
          tag = label
          tagpart = ""
      elseif tag == "" then
          tag = label
          tagpart = "[]"
      else
          tagpart = {"[", parse_inlines(tag), "]"}
      end
      if sps then
        tagpart = {sps, tagpart}
      end
      local r = references[normalize_tag(tag)]
      if r then
        return r
      else
        return nil, {"[", parse_inlines(label), "]", tagpart}
      end
  end

  -- lookup link reference and return a link, if the reference is found,
  -- or a bracketed label otherwise.
  local function indirect_link(label,sps,tag)
    return function()
      local r,fallback = lookup_reference(label,sps,tag)
      if r then
        return writer.link(parse_inlines_no_link(label), r.url, r.title)
      else
        return fallback
      end
    end
  end

  -- lookup image reference and return an image, if the reference is found,
  -- or a bracketed label otherwise.
  local function indirect_image(label,sps,tag)
    return function()
      local r,fallback = lookup_reference(label,sps,tag)
      if r then
        return writer.image(writer.string(label), r.url, r.title)
      else
        return {"!", fallback}
      end
    end
  end
%    \end{macrocode}
% \paragraph{Inline Element \acro{peg} Patterns}
%  \begin{macrocode}
  local Inline    = V("Inline")

  local Str       = normalchar^1 / writer.string

  local Ellipsis  = P("...") / writer.ellipsis

  local Smart     = Ellipsis

  local Symbol    = (specialchar - tightblocksep) / writer.string

  local Code      = inticks / writer.code

  local bqstart      = more
  local headerstart  = hash
                     + (line * (equal^1 + dash^1) * optionalspace * newline)

  if options.blankBeforeBlockquote then
    bqstart = fail
  end

  if options.blankBeforeHeading then
    headerstart = fail
  end

  local Endline   = newline * -( -- newline, but not before...
                        blankline -- paragraph break
                      + tightblocksep  -- nested list
                      + eof       -- end of document
                      + bqstart
                      + headerstart
                    ) * spacechar^0 / writer.space

%    \end{macrocode}
% Make two and more trailing spaces before a newline produce a forced line
% break, throw away one or more trailing spaces and an optional newline at the
% end of a file, and reduce one or more spaces and an optional newline into a
% single space.
%  \begin{macrocode}
  local Space     = spacechar^2 * Endline / writer.linebreak
                  + spacechar^1 * Endline^-1 * eof / ""
                  + spacechar^1 * Endline^-1 * optionalspace / writer.space

  -- parse many p between starter and ender
  local function between(p, starter, ender)
      local ender2 = B(nonspacechar) * ender
      return (starter * #nonspacechar * Ct(p * (p - ender2)^0) * ender2)
  end

  local Strong = ( between(Inline, doubleasterisks, doubleasterisks)
                 + between(Inline, doubleunderscores, doubleunderscores)
                 ) / writer.strong

  local Emph   = ( between(Inline, asterisk, asterisk)
                 + between(Inline, underscore, underscore)
                 ) / writer.emphasis

  local urlchar = anyescaped - newline - more

  local AutoLinkUrl   = less
                      * C(alphanumeric^1 * P("://") * urlchar^1)
                      * more
                      / function(url)
                        return writer.link(writer.string(url), url)
                      end

  local AutoLinkEmail = less
                      * C((alphanumeric + S("-._+"))^1 * P("@") * urlchar^1)
                      * more
                      / function(email)
                        return writer.link(writer.string(email),
                                           "mailto:"..email)
                      end

  local DirectLink    = (tag / parse_inlines_no_link)  -- no links inside links
                      * spnl
                      * lparent
                      * (url + Cc(""))  -- link can be empty [foo]()
                      * optionaltitle
                      * rparent
                      / writer.link

  local IndirectLink = tag * (C(spnl) * tag)^-1 / indirect_link

  -- parse a link or image (direct or indirect)
  local Link          = DirectLink + IndirectLink

  local DirectImage   = exclamation
                      * (tag / parse_inlines)
                      * spnl
                      * lparent
                      * (url + Cc(""))  -- link can be empty [foo]()
                      * optionaltitle
                      * rparent
                      / writer.image

  local IndirectImage  = exclamation * tag * (C(spnl) * tag)^-1 /
                         indirect_image

  local Image         = DirectImage + IndirectImage

  -- avoid parsing long strings of * or _ as emph/strong
  local UlOrStarLine  = asterisk^4 + underscore^4 / writer.string

  local EscapedChar   = S("\\") * C(escapable) / writer.string
%    \end{macrocode}
% \paragraph{Block Element \acro{peg} Patterns}
%  \begin{macrocode}
  local Block          = V("Block")

  local Verbatim       = Cs( (blanklines
                           * ((indentedline - blankline))^1)^1
                           ) / expandtabs / writer.verbatim

  -- strip off leading > and indents, and run through blocks
  local Blockquote     = Cs((
            ((leader * more * space^-1)/"" * linechar^0 * newline)^1
          * (-blankline * linechar^1 * newline)^0
          * blankline^0
          )^1) / parse_blocks / writer.blockquote

  local function lineof(c)
      return (leader * (P(c) * optionalspace)^3 * newline * blankline^1)
  end

  local HorizontalRule = ( lineof(asterisk)
                         + lineof(dash)
                         + lineof(underscore)
                         ) / writer.hrule

  local Reference      = define_reference_parser / register_link

  local Paragraph      = nonindentspace * Ct(Inline^1) * newline
                       * ( blankline^1
                         + #hash
                         + #(leader * more * space^-1)
                         )
                       / writer.paragraph

  local Plain          = nonindentspace * Ct(Inline^1) / writer.plain
%    \end{macrocode}
% \paragraph{List \acro{peg} Patterns}
%  \begin{macrocode}
  local starter = bullet + enumerator

  -- we use \001 as a separator between a tight list item and a
  -- nested list under it.
  local NestedList            = Cs((optionallyindentedline - starter)^1)
                              / function(a) return "\001"..a end

  local ListBlockLine         = optionallyindentedline
                                - blankline - (indent^-1 * starter)

  local ListBlock             = line * ListBlockLine^0

  local ListContinuationBlock = blanklines * (indent / "") * ListBlock

  local function TightListItem(starter)
      return -HorizontalRule
             * (Cs(starter / "" * ListBlock * NestedList^-1) /
                parse_blocks)
             * -(blanklines * indent)
  end

  local function LooseListItem(starter)
      return -HorizontalRule
             * Cs( starter / "" * ListBlock * Cc("\n")
               * (NestedList + ListContinuationBlock^0)
               * (blanklines / "\n\n")
               ) / parse_blocks
  end

  local BulletList = ( Ct(TightListItem(bullet)^1)
                       * Cc(true) * skipblanklines * -bullet
                     + Ct(LooseListItem(bullet)^1)
                       * Cc(false) * skipblanklines ) /
                         writer.bulletlist

  local function orderedlist(items,tight,startNumber)
    if options.startNumber then
      startNumber = tonumber(startNumber) or 1  -- fallback for '#'
    else
      startNumber = nil
    end
    return writer.orderedlist(items,tight,startNumber)
  end

  local OrderedList = Cg(enumerator, "listtype") *
                      ( Ct(TightListItem(Cb("listtype")) *
                           TightListItem(enumerator)^0)
                        * Cc(true) * skipblanklines * -enumerator
                      + Ct(LooseListItem(Cb("listtype")) *
                           LooseListItem(enumerator)^0)
                        * Cc(false) * skipblanklines
                      ) * Cb("listtype") / orderedlist

  local defstartchar = S("~:")
  local defstart     = ( defstartchar * #spacing * (tab + space^-3)
                     + space * defstartchar * #spacing * (tab + space^-2)
                     + space * space * defstartchar * #spacing *
                       (tab + space^-1)
                     + space * space * space * defstartchar * #spacing
                     )

  local dlchunk = Cs(line * (indentedline - blankline)^0)

  local function definition_list_item(term, defs, tight)
    return { term = parse_inlines(term), definitions = defs }
  end

  local DefinitionListItemLoose = C(line) * skipblanklines
                           * Ct((defstart *
                                 indented_blocks(dlchunk) /
                                 parse_blocks)^1)
                           * Cc(false)
                           / definition_list_item

  local DefinitionListItemTight = C(line)
                           * Ct((defstart * dlchunk /
                                            parse_blocks)^1)
                           * Cc(true)
                           / definition_list_item

  local DefinitionList =  ( Ct(DefinitionListItemLoose^1) * Cc(false)
                          +  Ct(DefinitionListItemTight^1)
                             * (skipblanklines *
                                -DefinitionListItemLoose * Cc(true))
                          ) / writer.definitionlist
%    \end{macrocode}
% \paragraph{Blank \acro{peg} Patterns}
%  \begin{macrocode}
  local Blank          = blankline / ""
                       + NoteBlock
                       + Reference
                       + (tightblocksep / "\n")
%    \end{macrocode}
% \paragraph{Heading \acro{peg} Patterns}
%  \begin{macrocode}
  -- parse Atx heading start and return level
  local HeadingStart = #hash * C(hash^-6) * -hash / length

  -- parse setext header ending and return level
  local HeadingLevel = equal^1 * Cc(1) + dash^1 * Cc(2)

  local function strip_atx_end(s)
    return s:gsub("[#%s]*\n$","")
  end

  -- parse atx header
  local AtxHeading = Cg(HeadingStart,"level")
                     * optionalspace
                     * (C(line) / strip_atx_end / parse_inlines)
                     * Cb("level")
                     / writer.heading

  -- parse setext header
  local SetextHeading = #(line * S("=-"))
                     * Ct(line / parse_inlines)
                     * HeadingLevel
                     * optionalspace * newline
                     / writer.heading
%    \end{macrocode}
% \paragraph{Top Level \acro{peg} Specification}
%  \begin{macrocode}
  syntax =
    { "Blocks",

      Blocks                = Blank^0 *
                              Block^-1 *
                              (Blank^0 / function()
                                return writer.interblocksep
                               end * Block)^0 *
                              Blank^0 *
                              eof,

      Blank                 = Blank,

      Block                 = V("Blockquote")
                            + V("Verbatim")
                            + V("HorizontalRule")
                            + V("BulletList")
                            + V("OrderedList")
                            + V("Heading")
                            + V("DefinitionList")
                            + V("Paragraph")
                            + V("Plain"),

      Blockquote            = Blockquote,
      Verbatim              = Verbatim,
      HorizontalRule        = HorizontalRule,
      BulletList            = BulletList,
      OrderedList           = OrderedList,
      Heading               = AtxHeading + SetextHeading,
      DefinitionList        = DefinitionList,
      DisplayHtml           = DisplayHtml,
      Paragraph             = Paragraph,
      Plain                 = Plain,

      Inline                = V("Str")
                            + V("Space")
                            + V("Endline")
                            + V("UlOrStarLine")
                            + V("Strong")
                            + V("Emph")
                            + V("NoteRef")
                            + V("Link")
                            + V("Image")
                            + V("Code")
                            + V("AutoLinkUrl")
                            + V("AutoLinkEmail")
                            + V("EscapedChar")
                            + V("Smart")
                            + V("Symbol"),

      Str                   = Str,
      Space                 = Space,
      Endline               = Endline,
      UlOrStarLine          = UlOrStarLine,
      Strong                = Strong,
      Emph                  = Emph,
      NoteRef               = NoteRef,
      Link                  = Link,
      Image                 = Image,
      Code                  = Code,
      AutoLinkUrl           = AutoLinkUrl,
      AutoLinkEmail         = AutoLinkEmail,
      InlineHtml            = InlineHtml,
      HtmlEntity            = HtmlEntity,
      EscapedChar           = EscapedChar,
      Smart                 = Smart,
      Symbol                = Symbol,
    }

  if not options.definitionLists then
    syntax.DefinitionList = fail
  end

  if not options.footnotes then
    syntax.NoteRef = fail
  end

  if not options.smartEllipses then
    syntax.Smart = fail
  end

  blocks = Ct(syntax)

  local inlines_t = util.table_copy(syntax)
  inlines_t[1] = "Inlines"
  inlines_t.Inlines = Inline^0 * (spacing^0 * eof / "")
  inlines = Ct(inlines_t)

  inlines_no_link_t = util.table_copy(inlines_t)
  inlines_no_link_t.Link = fail
  inlines_no_link = Ct(inlines_no_link_t)
%    \end{macrocode}
%
% \paragraph{Exported Conversion Function}
% Define \luamdef{reader->convert} as a function that converts markdown string
% \t`input` into a plain \TeX{} output and returns it. Note that the converter
% assumes that the input has \acro{unix} line endings.
%  \begin{macrocode}
  function self.convert(input)
    references = {}
%    \end{macrocode}
% When determining the name of the cache file, create salt for the hashing
% function out of the passed options recognized by the Lua interface (see
% Section \ref{sec:luaoptions}). The \Opt{cacheDir} option is disregarded.
%  \begin{macrocode}
    local opt_string = {}
    for k,_ in pairs(defaultOptions) do
      local v = options[k]
      if k ~= "cacheDir" then
        opt_string[#opt_string+1] = k .. "=" .. tostring(v)
      end
    end
    table.sort(opt_string)
    local salt = table.concat(opt_string, ",")
%    \end{macrocode}
% Produce the cache file, transform its filename via the \luam{writer->pack}
% method, and return the result.
%  \begin{macrocode}
    local name = util.cache(options.cacheDir, input, salt, function(input)
        return util.rope_to_string(parse_blocks(input)) .. writer.eof
      end, ".md" .. writer.suffix)
    return writer.pack(name)
  end
  return self
end
%    \end{macrocode}
% \subsubsection{Conversion from Markdown to Plain \TeX{}}
% The \luam{new} method returns the \luam{reader->convert} function of a reader
% object associated with the Lua interface options (see Section
% \ref{sec:luaoptions}) \t`options` and with a writer object associated with
% \t`options`.
%  \begin{macrocode}
function M.new(options)
  local writer = M.writer.new(options)
  local reader = M.reader.new(writer, options)
  return reader.convert
end

return M
%    \end{macrocode}
%
% \iffalse
%</lua>
%<*tex>
% \fi\subsection{Plain \TeX{} Implementation}\label{sec:teximplementation}
% The plain \TeX{} implementation provides macros for the interfacing between
% \TeX{} and Lua and for the buffering of input text. These macros are then
% used to implement the macros for the conversion from markdown to plain \TeX{}
% exposed by the plain \TeX{} interface (see Section \ref{sec:texinterface}).
%
% \subsubsection{Logging Facilities}\label{sec:texinterfacelogging}
%  \begin{macrocode}
\def\markdownInfo#1{%
  \message{(l.\the\inputlineno) markdown.tex info: #1.}}%
\def\markdownWarning#1{%
  \message{(l.\the\inputlineno) markdown.tex warning: #1}}%
\def\markdownError#1#2{%
  \errhelp{#2.}%
  \errmessage{(l.\the\inputlineno) markdown.tex error: #1}}%
%    \end{macrocode}
%
% \subsubsection{Options}
% The following definitions should be considered placeholder.
%  \begin{macrocode}
\def\markdownRendererLineBreakPrototype{\hfil\break}%
\let\markdownRendererEllipsisPrototype\dots
\long\def\markdownRendererCodeSpanPrototype#1{{\tt#1}}%
\long\def\markdownRendererLinkPrototype#1#2#3{#1}%
\long\def\markdownRendererImagePrototype#1#2#3{#1}%
\def\markdownRendererUlBeginPrototype{}%
\def\markdownRendererUlBeginTightPrototype{}%
\def\markdownRendererUlItemPrototype{}%
\def\markdownRendererUlEndPrototype{}%
\def\markdownRendererUlEndTightPrototype{}%
\def\markdownRendererOlBeginPrototype{}%
\def\markdownRendererOlBeginTightPrototype{}%
\def\markdownRendererOlItemPrototype{}%
\long\def\markdownRendererOlItemWithNumberPrototype#1{}%
\def\markdownRendererOlEndPrototype{}%
\def\markdownRendererOlEndTightPrototype{}%
\def\markdownRendererDlBeginPrototype{}%
\def\markdownRendererDlBeginTightPrototype{}%
\long\def\markdownRendererDlItemPrototype#1{#1}%
\def\markdownRendererDlEndPrototype{}%
\def\markdownRendererDlEndTightPrototype{}%
\long\def\markdownRendererEmphasisPrototype#1{{\it#1}}%
\long\def\markdownRendererStrongEmphasisPrototype#1{{\it#1}}%
\def\markdownRendererBlockQuoteBeginPrototype{\par\begingroup\it}%
\def\markdownRendererBlockQuoteEndPrototype{\endgroup\par}%
\long\def\markdownRendererInputVerbatimPrototype#1{%
  \par{\tt\input"#1"\relax}\par}%
\long\def\markdownRendererHeadingOnePrototype#1{#1}%
\long\def\markdownRendererHeadingTwoPrototype#1{#1}%
\long\def\markdownRendererHeadingThreePrototype#1{#1}%
\long\def\markdownRendererHeadingFourPrototype#1{#1}%
\long\def\markdownRendererHeadingFivePrototype#1{#1}%
\long\def\markdownRendererHeadingSixPrototype#1{#1}%
\def\markdownRendererHorizontalRulePrototype{}%
\long\def\markdownRendererFootnotePrototype#1{#1}%
%    \end{macrocode}
%
% \subsubsection{Lua Snippets}
% The \mdef{markdownLuaOptions} macro expands to a Lua table that
% contains the plain \TeX{} options (see Section \ref{sec:texoptions}) in a
% format recognized by Lua (see Section \ref{sec:luaoptions}). Note that the
% boolean options are not sanitized and expect the plain \TeX{} option macros
% to expand to either \t`true` or \t`false`.
%  \begin{macrocode}
\def\markdownLuaOptions{{%
  \ifx\markdownOptionBlankBeforeBlockquote\undefined\else
    blankBeforeBlockquote = \markdownOptionBlankBeforeBlockquote,
  \fi
  \ifx\markdownOptionBlankBeforeHeading\undefined\else
    blankBeforeHeading = \markdownOptionBlankBeforeHeading,
  \fi
  \ifx\markdownOptionCacheDir\undefined\else
    cacheDir = "\markdownOptionCacheDir",
  \fi
  \ifx\markdownOptionDefinitionLists\undefined\else
    definitionLists = \markdownOptionDefinitionLists,
  \fi
  \ifx\markdownOptionHashEnumerators\undefined\else
    hashEnumerators = \markdownOptionHashEnumerators,
  \fi
  \ifx\markdownOptionHybrid\undefined\else
    hybrid = \markdownOptionHybrid,
  \fi
  \ifx\markdownOptionFootnotes\undefined\else
    footnotes = \markdownOptionFootnotes,
  \fi
  \ifx\markdownOptionPreserveTabs\undefined\else
    preserveTabs = \markdownOptionPreserveTabs,
  \fi
  \ifx\markdownOptionSmartEllipses\undefined\else
    smartEllipses = \markdownOptionSmartEllipses,
  \fi
  \ifx\markdownOptionStartNumber\undefined\else
    startNumber = \markdownOptionStartNumber,
  \fi}
}%
%    \end{macrocode}
%
% The \mdef{markdownPrepare} macro contains the Lua code that is executed prior
% to any conversion from markdown to plain \TeX{}. It exposes the
% \luam{convert} function for the use by any further Lua code.
%  \begin{macrocode}
\def\markdownPrepare{%
%    \end{macrocode}
% First, ensure that the \m{markdownOptionCacheDir} directory exists.
%  \begin{macrocode}
  local lfs = require("lfs")
  local cacheDir = "\markdownOptionCacheDir"
  if lfs.isdir(cacheDir) == true then else
    assert(lfs.mkdir(cacheDir))
  end
%    \end{macrocode}
% Next, load the \t`markdown` module and create a converter function using
% the plain \TeX{} options, which were serialized to a Lua table via the
% \m{markdownLuaOptions} macro.
%  \begin{macrocode}
  local md = require("markdown")
  local convert = md.new(\markdownLuaOptions)
}%
%    \end{macrocode}
%
% \subsubsection{Lua \t`\textbackslash write18` Bridge}\label{sec:luabridge}
% The following \TeX{} code is intended for \TeX{} engines that do not provide
% direct access to Lua, but expose the shell of the operating system through
% the output file stream 18 (\Hologo{XeTeX}, \hologo{pdfLaTeX}).
% The \m{markdownLuaExecute} and \m{markdownReadAndConvert} macros defined here
% and in Section \ref{sec:directlua} are meant to be transparent to the
% remaining code.
%
% The package assumes that although the user is not using the Lua\TeX{} engine,
% their TeX distribution contains it, and uses shell access to produce and
% execute Lua scripts using the \TeX{}Lua interpreter (see
% \cite[Section~3.1.1]{luatex16}).
%  \begin{macrocode}
\ifx\directlua\undefined
%    \end{macrocode}
%
% The macro \mdef{markdownLuaExecuteFileStream} contains the number of the output
% file stream that will be used to store the helper Lua script in the file named
% \m{markdownOptionHelperScriptFileName} during the expansion of the macro
% \m{markdownLuaExecute}, and to store the markdown input in the file
% named \m{markdownOptionInputTempFileName} during the expansion of the macro
% \m{markdownReadAndConvert}.
%  \begin{macrocode}
  \csname newwrite\endcsname\markdownLuaExecuteFileStream
%    \end{macrocode}
%
% The \mdef{markdownLuaExecuteShellEscape} macro contains the numeric value of
% either the \m{pdfshellescape} (Lua\TeX{}, \Hologo{pdfTeX}) or the
% \m{shellescape} (\Hologo{XeTeX}) commands. This value indicates, whether the
% shell access is enabled (\t`1`), disabled (\t`0`), or restricted (\t`2`).  If
% neither of these commands is defined, act as if the shell access were enabled.
%  \begin{macrocode}
  \csname newcount\endcsname\markdownLuaExecuteShellEscape
  \ifx\pdfshellescape\undefined
    \ifx\shellescape\undefined
      \markdownLuaExecuteShellEscape=1%
    \else
      \markdownLuaExecuteShellEscape=\shellescape
    \fi
  \else
    \markdownLuaExecuteShellEscape=\pdfshellescape
  \fi
%    \end{macrocode}
%
% The \mdef{markdownLuaExecute} macro executes the Lua code it has received as
% its first argument. The Lua code may not directly interact with the \TeX{}
% engine, but it can use the \luam{print} function in the same manner it
% would use the \luam{tex.print} method.
%  \begin{macrocode}
  \def\markdownLuaExecute#1{%
%    \end{macrocode}
% If the shell is accessible, create the file
% \m{markdownOptionHelperScriptFileName} and fill it with the input Lua code
% prepended with \pkg{kpathsea} initialization, so that Lua modules from the
% \TeX{} distribution are available.
%  \begin{macrocode}
    \ifnum\markdownLuaExecuteShellEscape=1%
      \immediate\openout\markdownLuaExecuteFileStream=%
        \markdownOptionHelperScriptFileName
      \markdownInfo{Writing a helper Lua script to the file
        "\markdownOptionHelperScriptFileName"}%
      \immediate\write\markdownLuaExecuteFileStream{%
        local kpse = require('kpse')
        kpse.set_program_name('luatex') #1}%
      \immediate\closeout\markdownLuaExecuteFileStream
%    \end{macrocode}
% Execute the generated \m{markdownOptionHelperScriptFileName} Lua script using
% the \TeX{}Lua binary and store the output in the
% \m{markdownOptionOutputTempFileName} file.
%  \begin{macrocode}
      \markdownInfo{Executing a helper Lua script from the file
        "\markdownOptionHelperScriptFileName" and storing the result in the
        file "\markdownOptionOutputTempFileName"}%
      \immediate\write18{texlua "\markdownOptionHelperScriptFileName" >
        "\markdownOptionOutputTempFileName"}%
%    \end{macrocode}
% \m{input} the generated \m{markdownOptionOutputTempFileName} file.
%  \begin{macrocode}
      \input\markdownOptionOutputTempFileName\relax
    \else
%    \end{macrocode}
% If the shell is inaccessible, let the user know and suggest a remedy.
%  \begin{macrocode}
      \markdownError{I can not access the shell}{Either run the TeX
        compiler with the --shell-escape or the --enable-write18 flag,
        or set shell_escape=t in the texmf.cnf file}%
    \fi}%
%    \end{macrocode}
%
% The \mdef{markdownReadAndConvertTab} macro contains the tab character literal.
%  \begin{macrocode}
  \begingroup
    \catcode`\^^I=12%
    \gdef\markdownReadAndConvertTab{^^I}%
  \endgroup
%    \end{macrocode}
%
% The \m{markdownReadAndConvert} macro is largely a rewrite of the
% \Hologo{LaTeX2e} \m{filecontents} macro to plain \TeX{}.
%  \begin{macrocode}
  \begingroup
%    \end{macrocode}
% Make the newline and tab characters active and swap the character codes of the
% backslash symbol (\t`\textbackslash`) and the pipe symbol (\t`|`), so that
% we can use the backslash as an ordinary character inside the macro definition.
%  \begin{macrocode}
    \catcode`\^^M=13%
    \catcode`\^^I=13%
    \catcode`|=0%
    \catcode`\\=12%
    |gdef|markdownReadAndConvert#1#2{%
      |begingroup%
%    \end{macrocode}
% Open the \m{markdownOptionInputTempFileName} file for writing.
%  \begin{macrocode}
      |immediate|openout|markdownLuaExecuteFileStream%
        |markdownOptionInputTempFileName%
      |markdownInfo{Buffering markdown input into the temporary %
        input file "|markdownOptionInputTempFileName" and scanning %
        for the closing token sequence "#1"}%
%    \end{macrocode}
% Locally change the category of the special plain \TeX{} characters to
% \emph{other} in order to prevent unwanted interpretation of the input.
% Change also the category of the space and tab characters, so that we
% can retrieve them unaltered.
%  \begin{macrocode}
      |def|do##1{|catcode`##1=12}|dospecials%
      |catcode`| =12%
      |markdownMakeOther%
%    \end{macrocode}
% The \mdef{markdownReadAndConvertProcessLine} macro will process the individual
% lines of output. Note the use of the comments to ensure that the entire macro
% is at a single line and therefore no (active) newline symbols are produced.
%  \begin{macrocode}
      |def|markdownReadAndConvertProcessLine##1#1##2#1##3|relax{%
%    \end{macrocode}
% When the ending token sequence does not appear in the line, store the line in
% the \m{markdownOptionInputTempFileName} file.
%  \begin{macrocode}
        |ifx|relax##3|relax%
          |immediate|write|markdownLuaExecuteFileStream{##1}%
        |else%
%    \end{macrocode}
% When the ending token sequence appears in the line, make the next newline
% character close the \m{markdownOptionInputTempFileName} file, return the
% character categories back to the former state, convert the
% \m{markdownOptionInputTempFileName} file from markdown to plain \TeX{},
% \m{input} the result of the conversion, and expand the ending control
% sequence.
%  \begin{macrocode}
          |def^^M{%
            |markdownInfo{The ending token sequence was found}%
            |immediate|write|markdownLuaExecuteFileStream{}%
            |immediate|closeout|markdownLuaExecuteFileStream%
            |endgroup%
            |markdownInput|markdownOptionInputTempFileName%
            #2}%
        |fi%
%    \end{macrocode}
% Repeat with the next line.
%  \begin{macrocode}
        ^^M}%
%    \end{macrocode}
% Make the tab character active at expansion time and make it expand to a
% literal tab character.
%  \begin{macrocode}
      |catcode`|^^I=13%
      |def^^I{|markdownReadAndConvertTab}%
%    \end{macrocode}
% Make the newline character active at expansion time and make it consume the
% rest of the line on expansion. Throw away the rest of the first line and
% pass the second line to the \m{markdownReadAndConvertProcessLine} macro.
%  \begin{macrocode}
      |catcode`|^^M=13%
      |def^^M##1^^M{%
        |def^^M####1^^M{%
          |markdownReadAndConvertProcessLine####1#1#1|relax}%
        ^^M}%
      ^^M}%
%    \end{macrocode}
% Reset the character categories back to the former state.
%  \begin{macrocode}
  |endgroup
%    \end{macrocode}
%
% \subsubsection{Direct Lua Access}\label{sec:directlua}
% The following \TeX{} code is intended for \TeX{} engines that provide
% direct access to Lua (Lua\TeX{}). The \m{markdownLuaExecute} and
% \m{markdownReadAndConvert} defined here and in Section \ref{sec:luabridge}
% are meant to be transparent to the remaining code.
%  \begin{macrocode}
\else
%    \end{macrocode}
% The direct Lua access version of the \m{markdownLuaExecute} macro is defined
% in terms of the \m{directlua} primitive. The \luam{print} function is set as
% an alias to the \m{tex.print} method in order to mimic the behaviour of the
% \m{markdownLuaExecute} definition from Section \ref{sec:luabridge},
%  \begin{macrocode}
  \def\markdownLuaExecute#1{\directlua{local print = tex.print #1}}%
%    \end{macrocode}
%
% In the definition of the direct Lua access version of the
% \m{markdownReadAndConvert} macro, we will be using the hash symbol
% (\t`\#`), the underscore symbol (\t`_`), the caret symbol (\t`\^`), the
% dollar sign (\t`\$`), the backslash symbol (\t`\textbackslash`), the percent
% sign (\t`\%`), and the braces (\t`\{\}`) as a part of the Lua syntax.
%  \begin{macrocode}
  \begingroup
%    \end{macrocode}
%
% To this end, we will make the underscore symbol, the dollar sign, and caret
% symbols ordinary characters,
%  \begin{macrocode}
    \catcode`\_=12%
    \catcode`\$=12%
    \catcode`\^=12%
%    \end{macrocode}
% swap the category code of the hash symbol with the slash symbol (\t`/`).
%  \begin{macrocode}
    \catcode`\/=6%
    \catcode`\#=12%
%    \end{macrocode}
% swap the category code of the percent sign with the at symbol (\t`@`).
%  \begin{macrocode}
    \catcode`\@=14%
    \catcode`\%=12%
%    \end{macrocode}
% swap the category code of the backslash symbol with the pipe symbol (\t`|`),
%  \begin{macrocode}
    \catcode`|=0@
    \catcode`\\=12@
%    \end{macrocode}
% Braces are a part of the plain \TeX{} syntax, but they are not removed during
% expansion, so we do not need to bother with changing their category codes.
%  \begin{macrocode}
    |gdef|markdownReadAndConvert/1/2{@
%    \end{macrocode}
% Make the \mdef{markdownReadAndConvertAfter} macro store the token sequence
% that will be inserted into the document after the ending token sequence has
% been found.
%  \begin{macrocode}
      |def|markdownReadAndConvertAfter{/2}@
      |markdownInfo{Buffering markdown input and scanning for the
        closing token sequence "/1"}@
      |directlua{@
%    \end{macrocode}
% Set up an empty Lua table that will serve as our buffer.
%  \begin{macrocode}
        |markdownPrepare
        local buffer = {}
%    \end{macrocode}
% Create a regex that will match the ending input sequence. Escape any special
% regex characters (like a star inside \t`\textbackslash end\{markdown*\}`)
% inside the input.
%  \begin{macrocode}
        local ending_sequence = "^.-" .. ([[/1]]):gsub(
          "([%(%)%.%%%+%-%*%?%[%]%^%$])", "%%%1")
%    \end{macrocode}
% Register a callback that will notify you about new lines of input.
%  \begin{macrocode}
        |markdownLuaRegisterIBCallback{function(line)
%    \end{macrocode}
% When the ending token sequence appears on a line, unregister the callback,
% convert the contents of our buffer from markdown to plain \TeX{}, and insert
% the result into the input line buffer of \TeX{}.
%  \begin{macrocode}
          if line:match(ending_sequence) then
            |markdownLuaUnregisterIBCallback
            local input = table.concat(buffer, "\n") .. "\n\n"
            local output = convert(input)
            return [[\markdownInfo{The ending token sequence was found}]] ..
              output .. [[\markdownReadAndConvertAfter]]
%    \end{macrocode}
% When the ending token sequence does not appear on a line, store the line in
% our buffer, and insert either \m{fi}, if this is the first line of input,
% or an empty token list to the input line buffer of \TeX{}.
%  \begin{macrocode}
          else
            buffer[#buffer+1] = line
            return [[\]] .. (#buffer == 1 and "fi" or "relax")
          end
        end}}@
%    \end{macrocode}
% Insert \m{iffalse} after the \m{markdownReadAndConvert} macro in order to
% consume the rest of the first line of input.
%  \begin{macrocode}
      |iffalse}@
%    \end{macrocode}
% Reset the character categories back to the former state.
%  \begin{macrocode}
  |endgroup
\fi
%    \end{macrocode}
%
% \subsubsection{Typesetting Markdown}
% The \m{markdownInput} macro uses an implementation of the
% \m{markdownLuaExecute} macro to convert the contents of the file whose
% filename it has received as its single argument from markdown to plain
% \TeX{}.
%  \begin{macrocode}
\begingroup
%    \end{macrocode}
% Swap the category code of the backslash symbol and the pipe symbol, so that
% we may use the backslash symbol freely inside the Lua code.
%  \begin{macrocode}
  \catcode`|=0%
  \catcode`\\=12%
  |gdef|markdownInput#1{%
    |markdownInfo{Including markdown document "#1"}%
    |markdownLuaExecute{%
      |markdownPrepare
      local input = assert(io.open("#1","r")):read("*a") .. "\n\n"
%    \end{macrocode}
% Since the Lua converter expects \acro{unix} line endings, normalize the
% input.
%  \begin{macrocode}
      print(convert(input:gsub("\r\n?", "\n")))}}%
|endgroup
%    \end{macrocode}
%
% \iffalse
%</tex>
%<*latex>
% \fi\subsection{\LaTeX{} Implementation}\label{sec:lateximplementation}
% The \LaTeX{} implemenation makes use of the fact that, apart from some subtle
% differences, \LaTeX{} implements the majority of the plain \TeX{} format
% (see \cite[Section~9]{latex16}). As a consequence, we can directly reuse the
% existing plain \TeX{} implementation.
%  \begin{macrocode}
\input markdown
\ProvidesPackage{markdown}[\markdownVersion]%
%    \end{macrocode}
%
% \subsubsection{Logging Facilities}
% The \LaTeX{} implementation redefines the plain \TeX{} logging macros (see
% Section \ref{sec:texinterfacelogging}) to use the \LaTeX{} \m{PackageInfo},
% \m{PackageWarning}, and \m{PackageError} macros.
%  \begin{macrocode}
\renewcommand\markdownInfo[1]{\PackageInfo{markdown}{#1}}%
\renewcommand\markdownWarning[1]{\PackageWarning{markdown}{#1}}%
\renewcommand\markdownError[2]{\PackageError{markdown}{#1}{#2.}}%
%    \end{macrocode}
%
% \subsubsection{Typesetting Markdown}
% The \mdef{markdownInputPlainTeX} macro is used to store the original plain
% \TeX{} implementation of the \m{markdownInput} macro. The \m{markdownInput}
% is then redefined to accept an optional argument with options recognized by
% the \LaTeX{} interface (see Section \ref{sec:latexoptions}).
%  \begin{macrocode}
\let\markdownInputPlainTeX\markdownInput
\renewcommand\markdownInput[2][]{%
  \begingroup
    \markdownSetup{#1}%
    \markdownInputPlainTeX{#2}%
  \endgroup}%
%    \end{macrocode}
%
% The \env{markdown}, and \env{markdown*} \LaTeX{} environments are implemented
% using the \m{markdownReadAndConvert} macro.
%  \begin{macrocode}
\renewenvironment{markdown}{%
  \markdownReadAndConvert@markdown{}}\relax
\renewenvironment{markdown*}[1]{%
  \markdownSetup{#1}%
  \markdownReadAndConvert@markdown*}\relax
\begingroup
%    \end{macrocode}
% Locally swap the category code of the backslash symbol with the pipe symbol,
% and of the left (\t`\{`) and right brace (\t`\}`) with the less-than (\t`<`)
% and greater-than (\t`>`) signs. This is required in order that all the
% special symbols that appear in the first argument of the
% \t`markdownReadAndConvert` macro have the category code \emph{other}.
%  \begin{macrocode}
  \catcode`\|=0\catcode`\<=1\catcode`\>=2%
  \catcode`\\=12|catcode`|{=12|catcode`|}=12%
  |gdef|markdownReadAndConvert@markdown#1<%
    |markdownReadAndConvert<\end{markdown#1}>%
                           <|end<markdown#1>>>%
|endgroup
%    \end{macrocode}
%
% \subsubsection{Options}
% The supplied package options are processed using the \m{markdownSetup} macro.
%  \begin{macrocode}
\DeclareOption*{%
  \expandafter\markdownSetup\expandafter{\CurrentOption}}%
\ProcessOptions\relax
%    \end{macrocode}
%
% The following configuration should be considered placeholder.
%  \begin{macrocode}
\RequirePackage{url}
\RequirePackage{graphicx}
\RequirePackage{paralist}
\RequirePackage{fancyvrb}
\markdownSetup{rendererPrototypes={
  lineBreak = {\\},
  codeSpan = {\texttt{#1}},
  link = {#1\footnote{\ifx\empty#3\empty\else#3:
    \fi\texttt<\url{#2}\texttt>}},
  image = {\begin{figure}
      \begin{center}%
        \includegraphics{#2}%
      \end{center}%
      \ifx\empty#3\empty\else
        \caption{#3}%
      \fi
      \label{fig:#1}%
    \end{figure}},
  ulBegin = {\begin{itemize}},
  ulBeginTight = {\begin{compactitem}},
  ulItem = {\item},
  ulEnd = {\end{itemize}},
  ulEndTight = {\end{compactitem}},
  olBegin = {\begin{enumerate}},
  olBeginTight = {\begin{compactenum}},
  olItem = {\item},
  olItemWithNumber = {\item[#1.]},
  olEnd = {\end{enumerate}},
  olEndTight = {\end{compactenum}},
  dlBegin = {\begin{description}},
  dlBeginTight = {\begin{compactdesc}},
  dlItem = {\item[#1]},
  dlEnd = {\end{description}},
  dlEndTight = {\end{compactdesc}},
  emphasis = {\emph{#1}},
  strongEmphasis = {%
    \ifx\alert\undefined
      \textbf{\emph{#1}}%
    \else % Beamer support
      \alert{\emph{#1}}
    \fi},
  blockQuoteBegin = {\begin{quotation}},
  blockQuoteEnd = {\end{quotation}},
  inputVerbatim = {\VerbatimInput{#1}},
  horizontalRule = {\noindent\rule[0.5ex]{\linewidth}{1pt}},
  footnote = {\footnote{#1}}}}%

\ifx\chapter\undefined
  \markdownSetup{rendererPrototypes={
    headingOne = {\section{#1}},
    headingTwo = {\subsection{#1}},
    headingThree = {\subsubsection{#1}},
    headingFour = {\paragraph{#1}},
    headingFive = {\subparagraph{#1}}}}%
\else
  \markdownSetup{rendererPrototypes={
    headingOne = {\chapter{#1}},
    headingTwo = {\section{#1}},
    headingThree = {\subsection{#1}},
    headingFour = {\subsubsection{#1}},
    headingFive = {\paragraph{#1}},
    headingSix = {\subparagraph{#1}}}}%
\fi
%    \end{macrocode}
%
% \subsubsection{Miscellanea}
% Unlike base Lua\TeX{}, which only allows for a single registered function per
% a callback (see \cite[Section~8.1]{luatex16}), the \Hologo{LaTeX2e} format
% disables the \luam{callback.register} method and exposes the
% \luam{luatexbase.add_to_callback} and \luam{luatexbase.remove_from_callback}
% methods that enable the user code to hook several functions on a single
% callback (see \cite[Section~73.4]{latex16}).
%
% To make our code function with the \Hologo{LaTeX2e} format, we need to
% redefine the \m{markdownLuaRegisterIBCallback} and
% \m{markdownLuaUnregisterIBCallback} macros accordingly.
%  \begin{macrocode}
\renewcommand\markdownLuaRegisterIBCallback[1]{%
  luatexbase.add_to_callback("process_input_buffer", #1, %
    "The markdown input processor")}
\renewcommand\markdownLuaUnregisterIBCallback{%
  luatexbase.remove_from_callback("process_input_buffer",%
    "The markdown input processor")}
%    \end{macrocode}
%
% When buffering user input, we should disable the bytes with the high bit set,
% since these are made active by the \pkg{inputenc} package. We will do this by
% redefining the \m{markdownMakeOther} macro accordingly. The code is courtesy
% of Scott Pakin, the creator of the \pkg{filecontents} package.
%  \begin{macrocode}
\newcommand\markdownMakeOther{%
  \count0=128\relax
  \loop
    \catcode\count0=11\relax
    \advance\count0 by 1\relax
  \ifnum\count0<256\repeat}%
%    \end{macrocode}
%
% \iffalse
%</latex>
%<*context>
% \fi\subsection{\Hologo{ConTeXt} Implementation}
% \label{sec:contextimplementation}
% The \Hologo{ConTeXt} implementation makes use of the fact that, apart from
% some subtle differences, the Mark II and Mark IV \Hologo{ConTeXt} formats
% \emph{seem} to implement (the documentation is scarce) the majority of the
% plain \TeX{} format required by the plain \TeX{} implementation.  As a
% consequence, we can directly reuse the existing plain \TeX{} implementation
% after supplying the missing plain \TeX{} macros.
%  \begin{macrocode}
\def\dospecials{\do\ \do\\\do\{\do\}\do\$\do\&%
  \do\#\do\^\do\_\do\%\do\~}%
%    \end{macrocode}
% When there is no Lua support, then just load the plain \TeX{} implementation.
%  \begin{macrocode}
\ifx\directlua\undefined
  \input markdown
\else
%    \end{macrocode}
% When there is Lua support, check if we can set the \t`process_input_buffer`
% Lua\TeX{} callback.
%  \begin{macrocode}
  \directlua{%
    local function unescape(str)
      return (str:gsub("|", string.char(92))) end
    local old_callback = callback.find("process_input_buffer")
    callback.register("process_input_buffer", function() end)
    local new_callback = callback.find("process_input_buffer")
%    \end{macrocode}
%
% If we can not, we are probably using ConTeXt Mark IV. In ConTeXt Mark IV, the
% \t`process_input_buffer` callback is currently frozen (inaccessible from the
% user code) and, due to the lack of available documentation, it is unclear to
% me how to emulate it. Therefore, we will just force the plain \TeX{}
% implementation to use the \m{write18} bridge (see Section
% \ref{sec:luabridge}) by locally undefining the \m{directlua} primitive.
%  \begin{macrocode}
    if new_callback == false then
      tex.print(unescape([[|let|markdownDirectLua|directlua
                           |let|directlua|undefined
                           |input markdown
                           |let|directlua|markdownDirectLua
                           |let|markdownDirectLua|undefined]]))
%    \end{macrocode}
%
% If we can, then just load the plain \TeX{} implementation.
%  \begin{macrocode}
    else
      callback.register("process_input_buffer", old_callback)
      tex.print(unescape("|input markdown"))
    end}%
\fi
%    \end{macrocode}
%
% \subsubsection{Logging Facilities}
% The \Hologo{ConTeXt} implementation redefines the plain \TeX{} logging macros (see
% Section \ref{sec:texinterfacelogging}) to use the \Hologo{ConTeXt}
% \m{writestatus} macro.
%  \begin{macrocode}
\def\markdownInfo#1{\writestatus{markdown}{#1.}}%
\def\markdownWarning#1{\writestatus{markdown\space warn}{#1.}}%
%    \end{macrocode}
%
% \subsubsection{Typesetting Markdown}
% The \m{startmarkdown} and \m{stopmarkdown} macros are implemented using the
% \m{markdownReadAndConvert} macro.
%  \begin{macrocode}
\begingroup
%    \end{macrocode}
% Locally swap the category code of the backslash symbol with the pipe symbol.
% This is required in order that all the special symbols that appear in the
% first argument of the \t`markdownReadAndConvert` macro have the category code
% \emph{other}.
%  \begin{macrocode}
  \catcode`\|=0%
  \catcode`\\=12%
  |gdef|startmarkdown{%
    |markdownReadAndConvert{\stopmarkdown}%
                           {|stopmarkdown}}%
|endgroup
%    \end{macrocode}
%
% \subsubsection{Options}
% The following configuration should be considered placeholder.
%  \begin{macrocode}
\def\markdownRendererLineBreakPrototype{\blank}%
\long\def\markdownRendererLinkPrototype#1#2#3{%
  \useURL[#1][#2][][#3]#1\footnote[#1]{\ifx\empty#3\empty\else#3:
  \fi\tt<\hyphenatedurl{#2}>}}%
\long\def\markdownRendererImagePrototype#1#2#3{%
  \placefigure[][fig:#1]{#3}{\externalfigure[#2]}}%
\def\markdownRendererUlBeginPrototype{\startitemize}%
\def\markdownRendererUlBeginTightPrototype{\startitemize[packed]}%
\def\markdownRendererUlItemPrototype{\item}%
\def\markdownRendererUlEndPrototype{\stopitemize}%
\def\markdownRendererUlEndTightPrototype{\stopitemize}%
\def\markdownRendererOlBeginPrototype{\startitemize[n]}%
\def\markdownRendererOlBeginTightPrototype{\startitemize[packed,n]}%
\def\markdownRendererOlItemPrototype{\item}%
\long\def\markdownRendererOlItemWithNumberPrototype#1{\sym{#1.}}%
\def\markdownRendererOlEndPrototype{\stopitemize}%
\def\markdownRendererOlEndTightPrototype{\stopitemize}%
\definedescription
  [markdownConTeXtDlItemPrototype]
  [location=hanging,
   margin=standard,
   headstyle=bold]%
\definestartstop
  [MarkdownConTeXtDlPrototype]
  [before=\blank,
   after=\blank]%
\definestartstop
  [MarkdownConTeXtDlTightPrototype]
  [before=\blank\startpacked,
   after=\stoppacked\blank]%
\def\markdownRendererDlBeginPrototype{%
  \startMarkdownConTeXtDlPrototype}%
\def\markdownRendererDlBeginTightPrototype{%
  \startMarkdownConTeXtDlTightPrototype}%
\long\long\def\markdownRendererDlItemPrototype#1{%
  \markdownConTeXtDlItemPrototype{#1}}%
\def\markdownRendererDlEndPrototype{%
  \stopMarkdownConTeXtDlPrototype}%
\def\markdownRendererDlEndTightPrototype{%
  \stopMarkdownConTeXtDlTightPrototype}%
\long\def\markdownRendererEmphasisPrototype#1{{\em#1}}%
\long\def\markdownRendererStrongEmphasisPrototype#1{{\bf\em#1}}%
\def\markdownRendererBlockQuoteBeginPrototype{\startquotation}%
\def\markdownRendererBlockQuoteEndPrototype{\stopquotation}%
\long\def\markdownRendererInputVerbatimPrototype#1{\typefile{#1}}%
\long\def\markdownRendererHeadingOnePrototype#1{\chapter{#1}}%
\long\def\markdownRendererHeadingTwoPrototype#1{\section{#1}}%
\long\def\markdownRendererHeadingThreePrototype#1{\subsection{#1}}%
\long\def\markdownRendererHeadingFourPrototype#1{\subsubsection{#1}}%
\long\def\markdownRendererHeadingFivePrototype#1{\subsubsubsection{#1}}%
\long\def\markdownRendererHeadingSixPrototype#1{\subsubsubsubsection{#1}}%
\def\markdownRendererHorizontalRulePrototype{%
  \blackrule[height=1pt, width=\hsize]}%
\long\def\markdownRendererFootnotePrototype#1{\footnote{#1}}%
\stopmodule\protect
%    \end{macrocode}
%
% \iffalse
%</context>
% \fi
